\documentclass[a4paper, 11pt]{article}

% Math symbols, notation, etc.
% Apparently, must be loaded earlier than mathspec?
\usepackage{amsmath}
\usepackage{amssymb}
\usepackage{stmaryrd}
\usepackage{amsthm}

% Locale/encoding with XeTeX: UTF-8 by default, use fontspec
\usepackage{unicode-math}
\usepackage{polyglossia} % Modern replacement for Babel
\setmainlanguage{french} % or {english}
\usepackage{csquotes} % guillemets

% TeX Gyre Pagella = URW Palladio (free Palatino) extended
\setmainfont[Mapping=tex-text]{TeX Gyre Pagella}
\setmathfont{Asana Math}

% Other
\usepackage{fullpage}
%\usepackage{enumerate}
%\usepackage{graphicx}

% Macros de Jill-Jênn
\def\A{{\cal{A}}}
\def\F{\mathbb{F}}
\def\Z{\mathbb{Z}}
\def\N{\mathbb{N}}
\def\Q{\mathbb{Q}}
\def\U{\mathbb{U}}
\def\K{\mathbb{K}}
\def\P{\mathbb{P}}
\def\R{\mathbb{R}}
\def\C{\mathbb{C}}
\def\L{{\cal{L}}}
\def\S{{\cal{S}}}
\def\V{{\cal{V}}}
\def\T{{\cal{T}}}
\def\O{{\cal{O}}}
\def\Fs{{\cal{F}}}
\def\Ps{{\cal{P}}}
\def\Cf{{\cal{C}}}
\def\M{\mathcal{M}}
\def\MnK{{\cal M}_n(\K)}
\def\Tr{\textnormal{Tr}}
\def\Sp{\textnormal{Sp}}
\def\Re{\textnormal{Re}}
\def\Vect{\textnormal{Vect}}
\def\car{\textnormal{car}}
\def\pgcd{\textnormal{pgcd}}
\def\ppcm{\textnormal{ppcm}}
\def\Sigmap{\mathfrak{S}}
\def\prog{\texttt{prog}}

% Ajout personnel
\def\E{\mathbb{E}}
\def\Var{\textnormal{Var}}
\def\Ker{\textnormal{Ker}}
\def\Indic{\mathbb{1}}

\newtheorem*{definition}{Définition}
\newtheorem*{example}{Exemple}
\newtheorem*{proposition}{Proposition}
\newtheorem*{theorem}{Théorème}
\newtheorem*{application}{Application}
\newtheorem*{algo}{Algorithme}
\newtheorem*{lemma}{Lemme}
\newtheorem*{remark}{Remarque}
\newtheorem*{corollary}{Corollaire}

\begin{document}

\title{Mes développements d'agrég de math(-info)}
\author{Nguyễn Lê Thành Dũng}
\date{Préparation à l'agrégation de l'ENS Paris-Saclay, 2016--2017}
\maketitle

\tableofcontents

\section{Algèbre et analyse}

\subsection{Sous-espace de dimension finie de $\Cf(\R)$ stables par translation}

Classique.

\subsection{Fonction tangente et permutations zigzag}

Cf. Richard P. Stanley, \emph{A Survey of Alternating Permutations}. Le terme
\enquote{permutation alternante} pouvant induire en confusion avec le groupe
alterné, nous parlerons ici de \emph{permutation zigzag}.

\begin{theorem}[Désiré André]
La série génératrice des permutations haut-bas est $tan(z)+sec(z)$.
\end{theorem}

Merci à Paul Melotti, c'est fait dans son PDF.


\section{Algèbre}

\subsection{Points à distances impaires, avec le déterminant de Gram}

Le résultat suivant est tiré d'un petit article : \emph{Are there $n+2$ points
  in $E^n$ with odd integral distances}, Graham, Rothschild \& Straus. Il a été
posé comme exercice à l'oral de l'ENS Lyon en 2015.
\begin{theorem}
  Soit $n \in \N^*$. Il existe $n+2$ points distincts à distances entières
  impaires dans $\R^n$ si et seulement si $n+2 \equiv 0 \mod 16$.
\end{theorem}

La preuve originale utilise le \emph{déterminant de Cayley-Menger}, qui permet
de calculer le volume de simplexes et généralise ainsi la formule de Héron. On
trouvera des détails sur ce fameux déterminant dans le Zavidovique. Nous allons
utiliser ici le déterminant de Gram, plus connu, pour éviter des calculs avec
des opérations élémentaires sur les lignes et les colonnes.

\paragraph{Preuve du sens direct}

Soit $n \in \N$ et $x_0, \ldots, x_{n+1}$ des points dans $\R^n$ tels que pour
tout $i \neq j$, $\|x_i - x_j\| \in 2\Z + 1$. Quitte à translater, on peut
prendre $x_0 = 0$.

Notons $G = (\langle x_i, x_j \rangle)_{1 \leq i,j \leq n+1}$ la matrice de Gram
de $x_1, \ldots, x_{n+1}$, $J$ la matrice carrée de taille $n+1$ dont tous les
coefficients valent 1, et $A = (a_{ij})_{1 \leq i,j \leq n+1} = I+J$.

\begin{proposition}
  $\det G = 0$.
\end{proposition}
\begin{proof}
  Le déterminant de Gram d'une famille liée est toujours nul, et les $x_1,
  \ldots, x_{n+1}$ sont dans $\R^n$ donc sont liés. En effet, si $M$ est la
  matrice des $(x_i)$ dans une base orthonormale (de taille $n \times (n+1)$),
  alors $G = M^T M$ a un rang majoré par celui de $M$, qui est le rang de la
  famille de vecteurs.
\end{proof}

\begin{proposition}
  $\det A = n+2$.
\end{proposition}
\begin{proof}
  $J$ est de rang 1 donc a pour valeur propre 0 avec multiplicité $n$. On
  vérifie que $(1,\ldots,1)$ est vecteur propre pour la valeur propre $n+1$, ce
  qui achève de déterminer le spectre de $J$. Les valeurs propres de $A = I+J$
  sont donc $1, \ldots, 1, n+2$, leur produit vaut donc $n+2$.
\end{proof}

On va prouver $\det 2G \equiv \det A \mod 16$, ce qui suffira à conclure avec
les deux propositions ci-dessus. Pour cela, regardons ce qu'on peut dire sur
$2G$ modulo 16.

\begin{remark}
  Si $m$ est impair, $m^2 \equiv 1 \mod 8$ et $2m^2 \equiv 2 \mod 16$.
\end{remark}
\begin{proof}
  Facile à vérifier à la main ; le second résultat se déduit du premier, pas
  besoin d'examiner 8 classes de congruence.
\end{proof}

Ainsi, l'identité de polarisation $2 \langle x_i, x_j \rangle = \|x_i\|^2 +
\|x_j\|^2 - \|x_i - x_j\|^2$ entraîne que les coefficients non diagonaux de $2G$
sont congrus à 1 modulo 8 (remarquer que $\|x_i\|^2 = \|x_i - x_0\|^2$ est bien
le carré d'un nombre impair). Quant aux coefficients diagonaux, de la forme
$2\|x_i\|^2$, ils sont congrus à 2 modulo 16. Ainsi
\[ 2G \equiv A + B \mod 16, \qquad B = (b_{ij})_{1 \leq i,j \leq n+1} \in
  S_{n+1}(\Z),\; b_{ii} = 0,\; b_{ij} \in \{0,8\} \]
Donc $\det 2G \equiv \det(A+B) \mod 16$ car le déterminant est un polynôme à
coefficients entiers en les coefficients de la matrice. Reste à prouver
$\det(A+B) \equiv \det A \mod 16$.

Écrivons la formule de Leibniz :
\[ \det(A+B) = \sum_{\sigma \in \Sigmap_{n+1}} \varepsilon(\sigma) T_\sigma,
  \quad T_\sigma = \prod_{i=1}^{n+1} (a_{i \sigma(i)} + b_{i \sigma(i)}) \]
$A+B$ étant une matrice symétrique, $T_\sigma = T_{\sigma^{-1}}$ pour tout
$\sigma \in \Sigmap_n$ (et en général $\varepsilon(\sigma) =
\varepsilon(\sigma^{-1})$), et
\[ T_\sigma \equiv \prod_{i=1}^{n+1} a_{i \sigma(i)} \mod 8 \quad \text{donc}
  \quad T_\sigma + T_{\sigma^{-1}} = 2 T_\sigma \equiv 2 \prod_{i=1}^{n+1} a_{i
    \sigma(i)} \mod 16. \]

On peut ainsi regrouper les termes de la somme dans $\det(A+B)$ par deux pour
montrer la congruence voulue… à l'exception des termes pour $\sigma =
\sigma^{-1}$, c'est à dire $\sigma$ involutif.

Fixons $\sigma$ une involution et développons $T$. Étudions les termes
comportant au moins un facteur $b_i$. S'il y en a deux, alors le terme est
divisible par 64, donc congru à 0 mod 16. Donc
\[ T_\sigma \equiv \prod_{i=1}^{n+1} a_{i \sigma(i)} + \sum_{i=1}^{n+1} t_i
  \mod 16 \qquad t_i = b_{i \sigma(i)} \prod_{j \neq i} a_{j \sigma(j)}
  \equiv 0 \mod 8 \]
Pour $i \in \{1, \ldots, n+1\}$, alors :
\begin{itemize}
\item si $i$ est un point fixe de $\sigma$, alors $b_{i \sigma(i)} = b_{ii} = 0$
  donc $t_i = 0$ ;
\item sinon, $\sigma$ transpose $i$ et $\sigma(i)$, et $t_i = t_{\sigma(i)}$,
  toujours par symétrie des matrices $A$ et $B$.
\end{itemize}
Finalement, on peut toujours regrouper les termes non nuls par deux, et obtenir
que
\[ T_\sigma \equiv \prod_{i=1}^{n+1} a_{i \sigma(i)} \mod 16 \]
En fin de compte, on a :
\[ \det(A+B) \equiv \sum_{\sigma \in \Sigmap_{n+1}} \varepsilon(\sigma)
  \prod_{i=1}^{n+1} a_{i \sigma(i)} \equiv \det A \mod 16 \]
ce qu'il fallait démontrer.

\paragraph{Preuve du sens réciproque}

Soit $n = 16m - 2$, $m \in \N^*$. TODO.


\subsection{Constructibilité des polygones réguliers}

\[ \Q(\zeta_p) = K_m \supset K_{m-1} \supset \ldots \supset K_0 = \Q \]
\[ \{0\} \hookrightarrow \Z/2\Z \hookrightarrow \ldots \hookrightarrow \Z/2^n\Z
  \simeq (\Z/pZ)^* \simeq \textnormal{Gal}(\Q(\zeta_p)/\Q) \]

TODO. Utiliser un élément primitif pour borner le nombre d'automorphismes fixant
une extension intermédiaire d'un corps cyclotomique, ce qui évite d'avoir à
prouver un sens du lemme d'Artin.

\subsection{Automorphisme exceptionnel de $\Sigmap_6$}

Référence : à trouver (c'est dans H2G2 paraît-il).

On veut montrer que $\Sigmap_6$ admet un automorphisme extérieur\footnote{Ou
  encore que le groupoïde des ensembles de cardinal 6 admet un endofoncteur
  pleinement fidèle qui n'est pas naturellement isomorphe à l'identité.}.

\begin{remark}
  Il revient au même trouver un isomorphisme $\phi : \Sigmap_6 \to \Sigmap(E)$
  qui n'est induit par aucune bijection $\{1,\ldots,6\} \to E$.
\end{remark}
\begin{proof}
  En effet, si $E = \{1,\ldots,6\}$, l'isomorphisme (automorphisme, donc)
  $\Sigmap_6 \to \Sigmap_6$ induit par une permutation $\sigma$ de
  $\{1,\ldots,6\}$ est exactement la conjugaison par $\sigma$ (penser à l'action
  de la conjugaison sur la décomposition en cycles disjoints). On retrouve ainsi
  les automorphismes intérieurs.

  Réciproquement, si $\phi : \Sigmap_6 \to \Sigmap(E)$ n'est pas induit par une
  bijection, on a tout de même une égalité de cardinaux entraînant l'existence
  d'une bijection $f : \{1,\ldots,6\} \to E$, et alors quelle que soit $f$,
  $\phi \circ \Sigmap(f)^{-1}$ est un automorphisme extérieur. S'il s'agissait
  de la conjugaison par $\sigma$, on aurait $\phi = \Sigmap(f \circ \sigma)$
  (TODO vérifier si c'est pas $-1$).
\end{proof}

Nous allons construire une telle bijection à partir du lemme ci-dessous dont la
démonstration sera renvoyée à la fin.

\begin{lemma}
  Il existe un sous-groupe $H < \Sigmap_6$ d'indice 6 agissant transitivement
  sur $\{1,\ldots,6\}$ (pour l'action canonique de $\Sigmap_6$, bien entendu).
\end{lemma}

Posons maintenant $E = \Sigmap_6/H$ l'ensemble des classes à gauche. $\Sigmap_6$
agit dessus par translation ce qui donne un morphisme $\phi : \Sigmap_6 \to
\Sigmap(E)$. $\Ker(\phi)$ est distingué dans $\Sigmap_6$, et d'indice au moins 6
(taille d'une orbite), donc c'est $\{\textnormal{id}\}$. $\phi$ est ainsi
injective, et même bijective entre groupes de même cardinal.

Si $f$ était telle que $\phi = \Sigmap(f)$, alors le stabilisateur de $H \in E$
pour l'action $\phi$ serait le groupe des permutations ayant pour point fixe
$f^{-1}(H) \in \{1,\ldots,6\}$. Or ici, le stabilisateur de $H$ contient $H$, vu
comme sous-groupe de $\Sigmap_6$, qui agit transitivement donc dont les éléments
n'ont aucun point fixe commun. L'existence d'une telle $f$ est donc impossible.
$\phi$ est donc un isomorphisme qui n'est pas induit par une bijection.

\paragraph{Démonstration du lemme que nous avions délaissé} C'est assez bizarre
comme truc : on fait agir $\Sigmap_5$ sur ses 5-Sylow (au nombre de 6) par
conjugaison, c'est transitif (théorème de Sylow) et ça donne un morphisme
(injectif) de $\Sigmap_5$ dans le groupe des permutations de ces six 5-Sylow.
Sur un autre ensemble de 6 éléments (on avait dit $\{1,\ldots,6\}$ au départ),
il faut transférer par un choix arbitraire et non canonique de bijection. Ça
m'embête parce que c'est pas pratique pour s'assurer de la fonctorialité de la
procédure qui a un ensemble associe $E = \Sigmap_6/H$ correspondant.


\subsection{Réciprocité quadratique par les coniques}

Dans H2G2 tome 1.

\subsection{Théorème de l'élément primitif via le résultant}

(Cf. PDF de Paul Melotti.)

\begin{theorem}
  Si $L/K$ est une extension de corps finie séparable, alors il existe $\alpha
  \in L$ tel que $L = K(\alpha)$.
\end{theorem}

Comme on a \enquote{facilement} que $L = K(\alpha_1, \ldots, \alpha_n)$, on peut
faire une récurrence sur $n$ et il suffit pour cela de traiter le cas $n = 2$ :
$L = K(\alpha,\beta)$ avec $\alpha, \beta$ qui sont dans $L$ mais pas dans $K$.

Notons $P$ et $Q$ sont les polynômes minimaux respectifs de $\alpha$ et $\beta$
sur $K$. On cherche $\gamma$ tel que $L = K(\gamma)$, qu'on cherche de la forme
$\gamma = \alpha + t\beta$. 


\subsection{Dénombrement des polynômes irréductibles de $\F_q$}

Le résultat sur $\Z/p\Z$ permet de construire les corps finis comme corps de
rupture. Trouvable dans le poly de Pellerin. Utilise l'inversion de Möbius.

\section{Analyse}

\subsection{Les fonctions monotones sont dérivables presque partout}

Un joli résultat suggéré par le rapport de jury d'agrég, dont on peut notamment
déduire que la dérivation est bien l'opération inverse de l'intégrale de
Lebesgue (presque partout, évidemment). Ce qui suit est une tentative de
démonstration qui s'inspire princpalement des \emph{Leçons d'analyse
  fonctionnelle} de Riesz \& Szokefalvi-Nagy, la fin étant tirée de \emph{Real
  Analysis}, Royden (ou Royden \& Fitzpatrick pour la dernière édition). On
pourra aussi consulter \emph{An introduction to measure theory} de Terence Tao
pour plus de détails sur les théorèmes de dérivation presque partout.

Je ne suis pas du tout convaincu que ça rentre en 15 minutes… une approche
possible serait d'énoncer les lemmes sans démonstration (mais en dessinant
impérativement les rayons du soleil, et en agitant les mains), prouver
l'existence p.p. de $f'$ (le cœur de la démo), puis au choix prouver la finitude
p.p. ou l'un des lemmes.

Pour éviter le recours au lemme de recouvrement de Vitali ou de Besicovitch, on
va se contenter de montrer :
\begin{theorem}
  Soit $f : \R \to \R$ une fonction \textnormal{continue} croissante. Alors $f$ est
  dérivable presque partout.
\end{theorem}
L'hypothèse de continuité était d'ailleurs présente dans la démonstration de ce
théorème par Lebesgue en 1904.

Attention, la version générale ne s'en déduit pas immédiatement en se
restreignant à des intervalles de continuité ; reste le cas des \emph{fonctions
  de saut}, qui peuvent avoir des discontinuités denses. Exemple : $x \mapsto
\sum_{n \in \N} 2^{-n} \Indic_{[x \geq q_n]}$ où $(q_n)_{n \in \N}$ est une
énumération des rationnels. (Note : on sait qu'une fonction monotone a un nombre
au plus dénombrable de points de discontinuité.)

\begin{definition}
  On notera $\displaystyle \Delta f(x,y) = \frac{f(y) - f(x)}{y - x}$ pour $x
  \neq y$ et $f : \R \to \R$.
\end{definition}
Notation dont l'utilité ne fait aucun doute s'agissant de parler de dérivation
d'une fonction d'une variable réelle.

\paragraph{Résultats préliminaires} Commençons par deux petites propositions.
Pour $U \subseteq \R$ un ouvert, notons $CC(U)$ l'ensemble de ses composantes
connexes.
\begin{proposition}
  Les composantes connexes de $U$ sont des intervalles ouverts disjoints,
  en nombre au plus dénombrable, dont la réunion est égale à $U$.
\end{proposition}
\begin{proof}
  Admis, mais c'est classique et facile, donc à savoir démontrer sans hésiter.
\end{proof}
\begin{proposition}
  Soit $f : [a,b] \to \R$ croissante et $U \subset [a,b]$ ouvert dans $\R$.
  Alors
  \[ \sum_{]c,d[ \in CC(U)} (f(d)-f(c)) \leq f(b) - f(a) \]
  les termes de la somme étant positifs.
  Autrement dit, la \textnormal{variation totale} de $f$ sur $[a,b]$ est
  $f(b) - f(a)$.
\end{proposition}
\begin{proof}
  
  Supposons dans un premier temps $|CC(U)| < \infty$, soit $CC(U) =
  \{]c_1,d_1[,\ldots,]c_n,d_n[\}$ avec $c_1 < d_1 \leq c_2 < \ldots \leq d_n$.
  Alors pour tout $i$, $f(d_i) \leq f(c_{i+1})$ car $f$ croissante, donc
  \[ \sum_{i=1}^n (f(d_i) - f(c_i)) = - f(c_1) + (f(d_1) - f(c_2)) + \ldots +
    (f(d_{n-1}) - f(c_n)) + f(d_n) \leq f(d_n) - f(c_1) \]
  Dans le cas $CC(U)$ dénombrable, il suffit de passer à la limite sur les
  sous-familles finies.
\end{proof}

Le lemme qui suit est fondamental dans la preuve du théorème.

\begin{lemma}[des rayons du soleil]
  Soit $f : [a,b] \to \R$ continue et soit $\alpha \in \R$. Posons
  \[ O_g(]a,b[,\alpha)  = \left\{ x \in ]a,b[ | \exists y \in ]a,x[\,/\,
      \Delta f(x,y) < \alpha \right\} \]
  \[ O_d(]a,b[,\alpha) = \left\{ x \in ]a,b[ | \exists y \in ]x,b[\,/\,
      \Delta f(x,y) > \alpha \right\}
  \]
  Alors $O_g(]a,b[,\alpha)$ et $O_d(]a,b[,\alpha)$ sont ouverts dans $\R$ et
  \begin{enumerate}
  \item Pour tout $]c',d'[ \in CC(O_g(]a,b[,\alpha))$,
    $\displaystyle \Delta f(c,d) \leq \alpha$.
  \item Pour tout $]c,d[ \in CC(O_d(]a,b[,\alpha))$,
    $\displaystyle \Delta f(c,d) \geq \alpha$.
  \end{enumerate}
\end{lemma}
Il faut \textit{\textbf{faire un dessin}} pour voir ce qui se passe :
l'hypographe de $f$ est éclairé par un faisceau de rayons parallèles de pente
$\alpha$ provenant de la gauche dans le cas 1, la droite dans le cas 2, et les
ensembles définis sont les zones à l'ombre (la lumière passant à travers les
points de tangence). On peut constater visuellement qu'il y a en fait égalité
sauf éventuellement pour $d = b$ ou $c' = a$, mais nous n'aurons pas besoin de
le prouver formellement.

Remarquons que (1) découle de (2) appliqué à $x \mapsto -f(-x)$, et qu'on peut
se ramener à $\alpha = 0$ quitte à soustraire une fonction linéaire. (Et en
considérant $-f$, on peut obtenir deux autres cas, consistant à éclairer
l'épigraphe au lieu de l'hypographe.) Ce dernier cas est celui généralement
énoncé dans la littérature sous le nom de \enquote{rising sun lemma} (en effet,
$\alpha = 0$ revient à placer le soleil à l'horizon(tale), et les rayons
viennent de la droite i.e. de l'est).

\begin{proof}[Démonstration du lemme du soleil levant]
  $U$ est ouvert car les inégalités strictes de fonctions continues sont des
  inégalités ouvertes, et une projection linéaire d'un ouvert sur une coordonnée
  est un ouvert\footnote{Banach--Schauder en dimension finie !}.

  Soit $]c,d[ \in CC(U)$, on veut maintenant montrer que $f(d) - f(c) \geq 0$.
  Fixons $\varepsilon > 0$. Par compacité, $f$ atteint son maximum sur
  $[c+\varepsilon, d]$ en un point $x$. En particulier $f(x) \geq f(d)$. Or
  $f(d) \geq f(z)$ pour tout $z > d$ car $d \not\in U$. Ainsi $f(x) \geq f(y)$
  pour tout $y > x$, bref, aucun point ne peut faire de l'ombre à $x$ : $x
  \not\in U$. Or $[c+\varepsilon,d[ \subset U$, donc $x = d$. Autrement dit le
  maximum est atteint en l'unique point $d$, d'où $f(c+\varepsilon) < f(d)$, et
  en passant à la limite $f(c) \leq f(d)$.
\end{proof}

Ceci étant établi, c'est parti pour commencer à parler du théorème principal.

\paragraph{Stratégie d'attaque du théorème}
Soit $f : \R \to \R$ continue croissante. Définissons ses \emph{dérivées de Dini}
\[ D^+f(x) = \limsup_{y \to x^+} \Delta f(x,y) \qquad
  D^-f(x) = \limsup_{y \to x^-} \Delta f(x,y) \]
\[  D_+f(x) = \liminf_{y \to x^+} \Delta f(x,y) \qquad
  D_-f(x) = \liminf_{y \to x^+} \Delta f(x,y) \]

$f$ est dérivable en un point $x$ si et seulement si ces quatre limites (1)
coïncident et (2) sont finies. Nous allons établir que c'est le cas presque
partout. Mais d'abord, faisons le lien avec tout ce dont nous avons parlé avant.

\begin{remark}
  Soient $a < b$, $x \in ]a,b[$ et $\alpha < D_+f(x)$. Alors $x \in O_d(]a,b[,
  \alpha)$. De même, si $\alpha > D_-f(x)$, alors $x \in O_g(]a,b[, \alpha)$.
\end{remark}
\begin{proof}
  $\limsup_{y \to y^+} \Delta f(x,y) > \alpha$ et $]x,b[$ est un voisinage à
  droite de $x$ donc il existe $y \in ]x,b[$ tel que $\Delta(x,y) > \alpha$ :
  c'est exactement la condition d'appartenance à $O_d(]a,b[,\alpha)$.
\end{proof}

On dispose enfin de tous les outils pour attaquer le cœur de la preuve !

\begin{proof}[(1) Existence p.p. de la limite]
Montrons dans un premier temps que $D^+f \leq D_-f$ presque partout. Pour cela,
fixons $\alpha < \beta$ quelconques et posons
\[ S_{\alpha,\beta} = \{ x \in \R \mid D_-f(x) < \alpha < \beta < D^+f \} \]
On veut montrer que $S_{\alpha,\beta}$ est de mesure nulle.

Fixons un intervalle ouvert borné $E_0$ quelconque.
En partant de $E_0$, nous allons définir par récurrence deux suites de parties
de $\R$ : pour $n \in \N$,
\[ F_n = \bigcup_{I \in CC(E_n)} O_g(I, \alpha) \qquad
   E_{n+1} = \bigcup_{I \in CC(F_n)} O_d(I, \beta) \]
On a $E_0 \supset F_0 \supset E_1 \ldots$ et le lemme des rayons du soleil (à
l'aide une récurrence triviale) garantit que ce sont des ouverts. 

Si $x \in S_{\alpha,\beta} \cap E_0$, la remarque établie plus haut nous dit
que comme $D_-f(x) < \alpha$, $x \in F_0$, puis comme $D^+f(x) > \beta$, $x \in
E_1$, et ainsi de suite… Au bout du compte, $S_{\alpha,\beta} \cap E_0 \subset
\bigcap_{n \in \N} E_n$ et nous allons montrer que ce dernier ensemble est
négligeable.

Soit $n \in \N$. Soit $]a,b[ \in CC(F_n)$, le cas 1 du lemme nous dit que
$f(b) - f(a) \leq \alpha(b-a)$. Si maintenant $]c,d[ \in CC(E_{n+1} \cap
]a,b[)$, alors le cas 2 du lemme donne $f(d)-f(c) \geq \beta(d-c)$. En sommant
et en appliquant notre majoration de la variation totale, on a :
\[ \beta\mu(E_{n+1} \cap ]a,b[) = \sum_{]c,d[} \beta\mu(]c,d[) \leq \sum_{]c,d[}
  (f(d) - f(c)) \leq f(b) - f(a) \leq \alpha\mu(]a,b[) \]
\[ \mu(E_{n+1}) \leq \frac{\alpha}{\beta} \sum_{]a,b[} \mu(]a,b[) =
  \frac{\alpha}{\beta} \mu(F_n) \leq \frac{\alpha}{\beta} \mu(E_n) \] où $\mu$
est la mesure de Lebesgue. Comme $\alpha/\beta < 1$ et $E_0$ a été pris borné,
donc de mesure finie,
\[ \mu(S_{\alpha,\beta} \cap E_0) \leq \mu \left( \bigcap_{n=0}^\infty E_n \right)
  \leq \lim_{n \to \infty} \left( \frac{\alpha}{\beta} \right)^n \mu(E_0) = 0 \]
pour tout choix de $E_0$, d'où $\mu(S_{\alpha,\beta}) = 0$, puis
\[ \mu\left( \{x \in \R \mid  D^+f(x) > D_-f(x) \} \right)
  = \mu\left( \bigcup_{\alpha, \beta \in \Q} S_{\alpha,\beta} \right)
  = 0 \]
De façon analogue on peut montrer que $D^-f \leq D_+f$ p.p., ce qui suffit à
obtenir l'égalité p.p. des quatre dérivées de Dini : en effet, on sait que
$\liminf \leq \limsup$ donc $D_+f \leq D^+f$ et $D_-f \leq D^-f$.
\end{proof}

Notons $f'(x)$ cette limite commune qui existe pour tout $x$ hors d'un ensemble
de mesure nulle : on définit ainsi une fonction $f'$, qu'on étend arbitrairement
à $\R$. $f'$ est à valeurs dans $\R_+ \cup \{+\infty\}$. En effet, $f$ étant
croissante, ses taux d'accroissements sont positifs ; et rien ne garantit a
priori que la limite de ces taux d'accroissements soit finie. Reste à montrer :

\begin{proof}[(2) Finitude p.p. de la dérivée]
Calculons d'abord :
\[
  \int_a^b \frac{f(x+h) - f(x)}{h}dx =
    \frac{1}{h} \int_b^{b+h} f(x)\,dx - \frac{1}{h} \int_a^{a+h} f(x)\,dx
    \underset{h \to 0^+}{\longrightarrow} f(b) - f(a)
\]
où la dernière égalité est vraie par continuité de $f$. Comme $f'$ est mesurable
(en tant que limite de fonctions mesurables) et positive, on peut l'intégrer et
le lemme de Fatou nous donne
\[ \int_a^b f'(x)\,dx \leq \lim_{h \to 0^+} \int_a^b \frac{f(x+h) - f(x)}{h}dx =
  f(b)- f(a) < +\infty \] 
Ainsi, $f'$ est localement intégrale, elle est donc finie presque partout.
\end{proof}

\paragraph{Conclusion} En excluant d'abord les points où le taux d'accroissement
n'a pas de limite, puis ceux où la limite est $+\infty$, on ne s'est privé que
d'en semble de mesure nulle, donc : $f$ est dérivable presque partout !

Remarquons qu'en utilisant la majoration de la variation totale et le lemme des
rayons du soleil, on peut aussi démontrer (et ça a presque été fait dans la
preuve ci-dessus) que
\[ \mu\left( \left\{ x \in ]a,b[ \mid D^+f(x) \geq \beta \right\} \right)
  \leq \frac{f(b)-f(a)}{\beta} \]
ce qui aurait pu permettre de traiter la partie finitude (exercice pour la
lectrice). Ce résultat est à comparer avec le suivant :
\begin{theorem}[Inégalité maximale de Hardy-Littlewood]
  Soit $f \in L^1(\R)$, et soit $\beta > 0$. Alors
  \[ \mu\left( \left\{ x \in \R \mid \sup_{h > 0} \frac{1}{h} \int_x^{x+h}
        |f(t)|\,dt \geq \beta \right\} \right)
    \leq \frac{1}{\beta} \int_\R |f(t)|\,dt \]
\end{theorem}


\subsection{Théorème du minimax et dualité lagrangienne}

Un joli résultat venu du merveilleux monde de la recherche opérationnelle ! La
première partie est une généralisation d'un célèbre résultat de von Neumann en
théorie des jeux, la preuve étant tirée de \emph{Techniques of Variational
  Analysis} (TODO : auteurs). La seconde l'applique à l'optimisation convexe,
c'est du fait maison mais dont les techniques ne surprendront personne qui soit
habitué au domaine : on retrouve la relaxation lagrangienne et les conditions de
Karush--Kuhn--Tucker, mais ces dernières seraient normalement traitées dans le
cas général non convexe pour des extrema locaux dans un vrai cours
d'optimisation. Je me demande si on peut raccourcir les preuves ou affaiblir les
hypothèses pour cette dernière partie…

\paragraph{Théorème du minimax de Sion} TODO.


\paragraph{Application : dualité lagrangienne (cas convexe)} Considérons une
\emph{fonction objectif} $f \in \Cf^0(\R^n, \R)$ et des \emph{contraintes}
$g_1, \ldots, g_k \in \Cf^0(\R^n, \R)$, toutes convexes, définissant le problème
(primal) de minimisation
\[ p = \inf_{x \in F} f(x) \qquad F = \left\{ x \in \R^n \mid
  g_i(x) \leq 0,\; i = 1,\ldots,k \right\} \]
où $F$ est l'ensemble des \emph{solutions réalisables}.
Définissons le \emph{lagrangien} $\L : \R^n \times \R_+^k$ et le problème dual
associé par
\[ \L(x, \mu) = f(x) + \sum_{i=1}^k \mu_i g_i(x) \qquad
  d = \sup_{\mu \in \R_+^k} \inf_{x \in \R^n} \L(x,\mu) \]

On va supposer $d > -\infty$ ainsi que la \emph{condition de Slater} : il existe
$x_O \in \R^n$ \emph{strictement réalisable}, soit $g_i(x) < 0$ pour tout $i$.
Sous ces hypothèses, nous allons établir un théorème de dualité forte $p = d$,
ainsi que des conditions d'atteinte de l'optimum commun.

Fixons $R > 0$ et posons $\bar{B}_R = \bar{B}(x_0,R)$, qui est convexe compact.
Soit
\[ p_R = \min_{x \in \bar{B}_R} \sup_{\mu \in \R_+^k} \L(x,\mu)
       = \sup_{\mu \in \R_+^k} \min_{x \in \bar{B}_R} \L(x,\mu) \]
l'égalité résultant du théorème du minimax.
\begin{remark}
  À $x$ fixé, $\displaystyle \sup_{\mu \in \R_+^k} \L(x,\mu) = $ TODO typesetting.
\end{remark}
On en déduit que $p_R = \min_{x \in F \cap B_R} f(x)$ (l'ensemble étant non vide
puisque $x_0$ y est).

D'autre part, par la seconde expression, $p_R \geq d$. Comme $x_0$ est
strictement réalisable, quand $\|\mu\| \to +\infty$ en restant dans $\R_+^k$,
$\L(x_0, \mu) \to +\infty$ . Il existe donc un compact $K \subset \R_+^k$ tel
que $\forall \mu \in \R_+^k \setminus K,\; \L(x_0, \mu) < d \leq p_R$ (et $K$
est indépendant de $R$, c'est important !). D'où
\[ p_R = \max_{\mu \in K} \min_{x \in \bar{B}_R} \L(x,\mu) =
  \min_{x \in \bar{B}_R} \max_{\mu \in K} \L(x,\mu) \]
Quand $R \to +\infty$, on trouve
\[ \inf_{x \in F} f(x) = \inf_{x \in \R^n} \max_{\mu \in K} \L(x,\mu) 
  = \max_{\mu \in K} \inf_{x \in \R^n} \L(x,\mu)  \]
où, cette fois-ci, on a utilisé la compacité de $K$ pour appliquer le théorème
du minimax à $-\L$ ! Tout ceci fonctionnant encore avec n'importe quel convexe
compact $K' \supseteq K$, en épuisant $\R_+^k$ avec des compacts, on a
finalement $p = d$, et de plus, on sait qu'une solution duale optimale $\mu^*$
existe.

S'il reste du temps, on enchaîne avec la suite.

\paragraph{Caractérisation au premier ordre} On suppose maintenant que $f$ et
$g_1, \ldots, g_k$ sont $\Cf^1$, et on veut caractériser les solutions primales
optimales. Soit $x^*$ tel que $f(x^*) = p$. Alors
\[ f(x^*) = p = d \leq \L(x^*, \mu^*) = f(x^*) + \sum_{i=0}^k \mu^*_i
  g_i(x^*) \]
Comme $g_i(x^*) \leq 0$ et $\mu^*_i \geq 0$ pour tout $i$, ce qui entraîne
l'inégalité réciproque, on a en fait $\mu^*_i g_i(x^*) = 0$ pour tout $i$.
Ainsi, en $x^*$, $\L(-, \mu^*)$ atteint la valeur $d$ qui est son minimum, son
gradient s'annule donc. Soit :
\[ \nabla f(x^*) + \sum_{i=0}^k \mu^*_i \nabla g_i(x^*) = 0 \]
et on constate que les $\mu_i$ sont des \emph{multiplicateurs de Lagrange} pour
des contraintes d'\emph{inégalité} !

Réciproquement, soient $x \in F$ et $\mu \in R_+^k$ vérifiant cette condition du
premier ordre ainsi que $\mu_i g_i(x) = 0$ pour tout $i$. Alors, comme
$\L(-,\mu)$ est convexe, ses points stationnaires sont ses minima globaux, donc
elle atteint son minimum en $x$. On peut vérifier, grâce à la seconde condition,
que cette valeur minimale est $f(x)$ ; puis on en déduit que $x$ est optimal
pour le primal et $\mu$ est optimal pour le dual.

TODO : voir si on peut pas boucher les trous
dans la disjonction de cas de la dualité forte, avec un argument d'hyperplan
séparateur.



\subsection{Théorème du point fixe de Schauder}

\begin{theorem}
  Soit $K$ une partie compacte convexe d'un EVN, $f : K \to K$ continue. Alors
  $f$ admet au moins un point fixe.
\end{theorem}
Application : théorème d'existence de Cauchy-Peano.


\subsection{Récurrence d'une marche aléatoire via séries de Fourier}

Un théorème célèbre, avec une preuve étrangement moins célèbre et pourtant
stylée, que j'ai découverte dans le cours de processus aléatoires de Josselin
Garnier et qui est également trouvable sur Internet.

\begin{theorem}[Pólya]
  La marche aléatoire symétrique sur $\Z^d$ est \emph{récurrente} si et
  seulement si $d \leq 2$.
\end{theorem}

Précisons ce que signifie \enquote{récurrente}. Cette marche aléatoire est
définie par la suite de v.a. $S_n = X_1 + \ldots + X_n$ où les $(X_i)$ sont iid
uniformes parmi $\{\pm e_1, \ldots, \pm e_d\}$. Soit $N$ le nombre de passages à
l'origine : $N = \mathrm{Card}\left\{n \in \N \mid S_n = 0\right\}$. On dit que
la marche est récurrente quand $\E[N] = +\infty$.


\paragraph{Premières remarques} On a $\displaystyle \E[N] = \E\left[
  \sum_{n=0}^\infty \Indic_{\{S_n = 0\}} \right] =
\sum_{n=0}^\infty \P(S_n = 0)$.\\
On peut ne garder que les termes pairs dans cette somme. En effet, pour tout $n
\in \N$, on établit facilement que $\sum_{i=1}^d \langle S_n, e_i \rangle \equiv
n \mod 2$, ce qui entraîne que $S_n \neq 0$ quand $n$ est impair.\\
On cherche donc une expression pour $\P(S_n = 0)$, $n = 2m$ avec $m \in \N$.

\paragraph{Utilisation de la fonction caractéristique}

Si $\varphi$ est la fonction caractéristique de $X_1$, alors celle de $S_n$
est égale à $\varphi^n$ (somme de v.a. indépendantes).

\[ \forall x \in \R^d,\quad \varphi(x) = \E\left[ e^{i \langle x, X_1 \rangle}
  \right]
  = \sum_{j=1}^d \left( \frac{1}{2d}e^{ix_j} + \frac{1}{2d}e^{-ix_j}\right)
  = \frac{1}{d} \sum_{j=1}^d \cos(x_i) \]

Pour en déduire les probabilités recherchées on utilise :
\begin{lemma} Soit $k = (k_1, \ldots, k_d) \in \Z^d$. On a la formule des
  coefficients de Fourier :
 \[ \displaystyle \P(S_n = k) = \frac{1}{(2\pi)^d} \int_{[-\pi,\pi]^d} \varphi_n(x)e^{-i\langle k,x \rangle} \]
\end{lemma}
En fait, $\phi_n$ est une série de Fourier $d$-dimensionnelle normalement
convergente dont les coefficients sont les $\P(S_n = (k_1, \ldots, k_d))$. Ce
n'est pas anecdotique : la preuve qu'on est en train de dérouler repose
fondamentalement sur un passage au domaine de Fourier\footnote{Ici les
  \enquote{fréquences} (le dual) sont dans le tore et le \enquote{temps} (le
  primal) est discret ; on est habitués à l'autre sens, mais c'est juste
  l'involutivité de la dualité de Pontriaguine. Les anglophones parlent de
  \enquote{discrete-time Fourier transform}, à ne pas confondre avec la
  transformée de Fourier discrète d'un signal \emph{fini}.} pour convertir une
convolution de mesures en produit de fonctions, c'est complètement dans l'esprit
de l'analyse de Fourier. Mais ici, pas besoin de toute cette théorie, il suffit
de calculer :
\begin{proof}[Preuve du lemme]
  \[ \int_{[-\pi,\pi]^d} \varphi_n(x)e^{-i\langle k,x \rangle} dx =
    \int_{[-\pi,\pi]^d} \E\left[ e^{i\langle S_n, x \rangle} e^{-i\langle k,x
        \rangle}\right] dx
    = \E\left[ \int_{[-\pi,\pi]^d} e^{i\langle S_n - k, x \rangle} dx \right]
    = \E\left[ (2\pi)^d \Indic_{\{S_n = k\}} \right] \]
  où on a pu intervenir $\int$ et $\E$ grâce au théorème de Fubini appliqué à
  une fonction bornée, donc intégrable sur un produit d'espaces de mesure finie.
\end{proof}
Ainsi, en se souvenant que $\varphi$ est à valeurs réelles, ses puissances
paires sont positives (!), et
\[ (2\pi)^d \E[N] = \sum_{m=0}^{\infty} \int_{[-\pi,\pi]^d} \varphi(x)^{2m} dx
  = \int_{[-\pi,\pi]^d} \frac{dx}{1 - \varphi(x)^2}\]
par convergence monotone, l'expression finale étant valable parce que
$|\varphi(x)| < 1$ presque partout. (Si on avait gardé les termes impairs (qui
peuvent être négatifs), comme c'est le cas dans certaines références,
l'interversion série-intégrale aurait été plus dure à justifier ; il faut une
astuce dans ce cas…)

\paragraph{Étude d'une intégrale et conclusion} Il s'agit donc de savoir si
cette intégrale est finie ou non. $1/(1-\varphi^2)$ part à l'infini exactement
aux points où il y a un problème de convergence à savoir $|\varphi| = 1$ , soit
$(0,\ldots,0)$ ainsi que les $2^d$ points $(\pm \pi, \ldots, \pm \pi)$ ; et elle
est continue en-dehors de ces points. Ces derniers points sont des anti-périodes
de $\varphi$, donc des périodes de $1/(1-\varphi^2)$ : il suffit donc d'étudier
l'intégrabilité en 0, sur un voisinage $B(0,\varepsilon)$ où $0 < \varepsilon <
\pi$.

Pour $x \to 0$,
\[ 1 - \varphi(x) = \frac{1}{d} \sum_{j=1}^d (1 - \cos x_j) \sim
  -\frac{1}{d}(x_1^2 + \ldots + x_d^2) \] d'où
\[\frac{1}{1 - \varphi(x)^2} = \frac{1}{(1 + \varphi(x))(1 - \varphi(x))} \sim
  \frac{2d}{\|x\|^2} \]
Finalement, reste à étudier l'intégrabilité de $\|x\|^{-2}$. Un passage en
coordonnées polaires et le théorème de Tonelli donnent
\[ \int_{B(0,\varepsilon)} \|x\|^{-2}\,dx = \int_{]0,\varepsilon[ \times S^{d-1}}
  r^{-2} \cdot r^{d-1}\,dr\,d\omega = \textnormal{Vol}(S^{d-1}) \int_0^\varepsilon
  r^{d-3}\,dr \]
Ce qui est infini exactement quand $d < 3$, CQFD.

\paragraph{Si le temps permet…} On expliquera pourquoi dans une chaîne de Markov
irréductible, un état est récurrent si et seulement si tous le sont, et que dans
ce cas, tous les états sont presque sûrement atteints. Ainsi, on a montré qu'un
ivrogne qui se déplace dans le plan peut être sûr de rentrer chez lui.


\subsection{Théorème de bornitude uniforme et série de Fourier divergente}

Moralement : l'évaluation en 0 de la somme partielle de Fourier est un produit
scalaire avec un noyau de Dirichlet, dont la norme tend vers $+\infty$.

\subsection{Inversion de Fourier et théorème de continuité de Lévy (en dimension 1)}

Notre but est de montrer, dans le cas $d = 1$ (mais toutes les preuves se
généralisent sans mal à $d$ quelconque) :

\begin{theorem}[Inversion de Fourier]
  Soit $f \in L^1(\R^d)$. Si $\hat{f} \in L^1(\R^d)$, on a pour presque
  tout $x \in \R^d$ :
\[ f(x) = \frac{1}{(2\pi)^d} \int_\R \hat{f}(\xi) e^{i\xi \cdot x} \,d\xi =
  \hat{\hat{f}}(-x). \]
\end{theorem}

En soi, la démonstration est un peu courte, donc on combine généralement ça avec
le calcul de la transformée de Fourier d'une gaussienne (résultat utilisé dans
la preuve). C'est utile pour faire rentrer le développement dans la leçon
\enquote{Illustrer par des exemples quelques méthodes de calcul d'intégrales …},
mais pour les autres circonstances, on propose ici un autre complément (suggéré
par N. Clozeau) : appliquer la formule d'inversion pour prouver le théorème de
continuité de Lévy (sur une variable aléatoire dans $\R$ quelconque, pas
forcément à densité !).

\paragraph{Préliminaires sur les gaussiennes}
On définit la gaussienne d'écart-type $\sigma > 0$ comme $g_\sigma : x \mapsto
(\sigma\sqrt{2\pi})^{-1}\exp(-x^2/2\sigma^2)$. On a $g_\sigma \in \S(\R)$, donc
elle est intégrable tout comme toutes ses dérivées.

\begin{lemma}
  Pour tout $f \in L^1(\R)$,
  $g_\sigma * f \underset{\sigma \to 0}{\longrightarrow} f$ dans $L^1(\R)$.
\end{lemma}
\begin{proof}
  Admis. Cette propriété découle du fait que les gaussiennes sont des
  approximations de l'unité.
\end{proof}

Attention, convergence $L^p$ n'entraîne pas convergence presque partout ! Par
contre, d'une suite qui converge dans $L^p$, on peut extraire une sous-suite qui
converge vers la même limite presque partout, d'où :
\begin{corollary}
  Il existe une suite $\sigma_n$ décroissante tendant vers 0 telle que
  $g_{\sigma_n} * f \rightarrow f$ p.p.
\end{corollary}
(Rappel : toute série absolument convergente dans $L^p$ converge p.p., ça
sert dans la preuve de complétude.)

\begin{proposition}[Transformée de Fourier des gaussiennes]
  $\displaystyle \widehat{g_\sigma}(\xi) =
  \exp\left(- \frac{\sigma^2 \xi^2}{2} \right) =
  \frac{\sqrt{2\pi}}{\sigma} g_{1/\sigma}(\xi)$.
\end{proposition}
\begin{proof}
  Il y en a plein, choisissez celle qui vous plaît. Une façon simple est
  d'utiliser le théorème de dérivation sous le signe intégral pour montrer que
  c'est une solution de l'équation différentielle $y' = -\sigma^2xy $. Comme
  $\widehat{g_\sigma}(0) = 1$ on peut parachuter la solution $\sqrt{2\pi}/\sigma
  \cdot g_{1/\sigma}$, Cauchy-Lipschitz linéaire assurant l'unicité.
\end{proof}

En itérant deux fois, on a bien $\widehat{\widehat{g_{\sigma}}} = 2\pi g_\sigma$
(le signe moins a disparu car $g_\sigma$ est paire) : la formule d'inversion de
Fourier est vérifiée par les gaussiennes.

\paragraph{Inversion de Fourier, cas général}

Soit $f \in L^1(\R)$ telle que $\hat{f} \in L^1(\R)$. On pose
\[ F_\sigma(x) = \int_\R \widehat{g_\sigma}(\xi)\hat{f}(\xi)e^{i\xi x} d\xi \]
Un calcul rapide (on écrit la formule intégrale pour $\hat{f}$, puis on utilise
Fubini) entraîne que
\[ F_\sigma(x) = \widehat{\widehat{g_\sigma}} * f(\xi) \]

On observe que $\widehat{g_\sigma} \longrightarrow 1$ ponctuellement quand
$\sigma \to 0$ ; en passant à la limite sur la suite $\sigma_n$ du lemme
préliminaire, on trouve l'égalité presque partout de la formule d'inversion de
Fourier. Pour ce faire on a besoin crucialement que $\hat{f} \in L^1(\R)$ pour
que le théorème de convergence dominée s'applique.

\paragraph{Application : théorème de continuité de Lévy} Rappelons tout d'abord
que pour vérifier une convergence en loi, il suffit de tester sur un ensemble de
fonctions dont l'adhérence contient $\Cf_c(\R)$, espace de fonctions continues
à support compact, alors que la définition de cette convergence fait intervenir
les fonctions continues bornées. C'est moralement un argument de densité mais
attention, $\Cf_c(\R)$ n'est pas dense dans $\Cf_b(\R)$ pour la norme uniforme !

\begin{theorem}[Lévy]
  Soit $X$ une variable aléatoire réelle de fonction caractéristique $\phi$, et
  $(X_n)_{n \in \N}$ une suite de v.a. réelles dont on notera les fonctions
  caractéristiques $\phi_n$ pour $n \in \N$. Alors $X_n \rightarrow X$ en loi si
  et seulement si $\phi_n \rightarrow \phi$ simplement.
\end{theorem}
\begin{proof}
  Le sens direct est évident, prouvons la réciproque. Soit $f \in
  \Cf^\infty_c(\R)$ (qui est bien dense dans $\Cf_c(\R)$), alors en particulier
  $f \in \S(\R)$ donc $f$ et $\hat{f}$ sont intégrables. Par conséquent, on peut
  écrire la formule d'inversion de Fourier sur $f$, puis prendre une espérance :
  \[ \forall n \in \N,\; \E[ f(X_n) ] = 
    \E\left[ \frac{1}{2\pi} \int_\R \hat{f}(\xi) e^{i\xi X_n}\,d\xi \right] =
    \frac{1}{2\pi} \int_\R \E\left[ \hat{f}(\xi) e^{i\xi X_n} \right] d\xi =
    \frac{1}{2\pi} \int_\R \hat{f}(\xi) \phi_n(\xi)\, d\xi
  \]
  où l'on a interverti espérance et intégrale par le théorème de Fubini (à
  justifier). (Au fond, on n'a fait que constater que la transformation de
  Fourier préserve le produit scalaire.) On peut faire le même calcul pour
  $\E[f(X)]$, puis comme $|\phi_n| \leq 1$ et $\hat{f} \in L^1(\R)$, le théorème
  de convergence dominée et notre  hypothèse de convergence des $\phi_n$ nous
  donnent
  \[ \E[ f(X_n) ] \underset{n \to \infty}{\longrightarrow}
    \frac{1}{2\pi} \int_\R \hat{f}(\xi) \phi(\xi)\, d\xi = \E[f(X)].  \]
\end{proof}


\subsection{Polynômes de Bernstein}

Référence : Zuily-Quéffélec.

\begin{theorem}[Bernstein]
  Soit $f \in \Cf([0,1], \R)$.
  
  Pour $n \in \N$ et $x \in \R$, on pose $B_n(x) = \E[f(X)]$ où $X \sim
  \textnormal{Binom}(n,x)$. Alors :
  \begin{enumerate}
  \item Pour tout $n \in \N$, $B_n : [0,1] \to \R$ est une fonction polynomiale.
    ($B_n$ est appelé le $n$-ième polynôme de Bernstein de $f$.)
  \item $\|B_n - f\|_\infty \underset{n \to \infty}{=} O\left(
      \omega_f(n^{-1/2}) \right)$.
  \end{enumerate}
\end{theorem}

On rappelle que
\[ \omega_f(h) = \sup_{|x-y| \leq h} |f(x) - f(y)| \]
est le \emph{module de continuité uniforme} de $f$.

\begin{corollary}[Weierstrass]
  Les fonctions polynomiales sont denses dans $\Cf([0,1])$.
\end{corollary}

Le corollaire découle directement du théorème de Heine, le point 1 du théorème
est trivial en écrivant la formule de l'espérance.

Démonstration du théorème :
\begin{enumerate}
\item À $x$ fixé, écrire $|B_n(x) - f(x)|$ comme une espérance, qu'on majore en
  utilisant $\omega_f$
\item Astuce : utiliser $\omega_f(\lambda h) \leq \lceil \lambda \rceil
  \omega_f(h)$ pour obtenir $\omega_f(n^{-1/2})$ fois un facteur probabiliste
\item Majorer l'espérance de ce facteur en faisant intervenir la variance
  (dépendante de $x$, paramètre de la loi binomiale)…
\item puis indépendamment de $x$, et c'est fini !
\end{enumerate}

TODO : optimalité de la borne par le théorème central limite.

Il faut bien connaître le théorème de Stone-Weierstrass plus général pour
répondre aux questions sur le développement.

\subsection{Méthode de Newton (lol)}

\subsection{Lemme de Morse}

Pour avoir une application aux formules de Taylor.

\section{Informatique}

\subsection{Minimisation par double renversement (avec complexité)}

Source : Sakarovitch.

Notons $D$ la déterminisation d'un automate, prenant en entrée un AFN et
renvoyant un AFDC, partie accessible de l'automate des parties. Notons $T$ la
transposition d'un automate, obtenue en renversant le sens des flèches. La
transposée d'un automate reconnaissant $L$ reconnaît $\tilde{L}$, langage miroir
de $L$.

\begin{algo}[Calcul d'un AFDC minimal à partir d'un AFN]
  Si $\mathcal{A}$ est un automate reconnaissant le langage $L$, $D \circ T
  \circ D \circ T(\mathcal{A})$ est un AFDC minimal reconnaissant $L$.
\end{algo}

Comme la dernière opération est une déterminisation, la sortie est bien un AFDC.
Le langage reconnu est préservé puisque $D$ le préserve et $T$ prend le miroir,
ce qui est involutif. Reste à prouver la minimalité.

\begin{lemma}[Brzozowski]
  Soit $\mathcal{B}$ un automate reconnaissant $L$, \emph{co-déterministe} et
  \emph{co-accessible} (autrement dit, dont la transposée est un AFD
  accessible). Soient $u$ et $v$ vérifiant $u^{-1}L = v^{-1}L$. Alors tout état
  atteignable en lisant $u$ l'est en lisant $v$.
\end{lemma}
\begin{proof}
  Soit $q$ atteignable à partir de $u$, $q$ est co-accessible donc il existe $w$
  atteignant l'unique état final à partir de $q$. $uw \in L$ donc $vw \in L$, et
  le seul chemin acceptant pour $vw$ doit passer par $q$ après avoir lu $v$ par
  co-déterminisime… Faire un dessin !
\end{proof}

Appliquons maintenant le lemme à $\mathcal{B} = T \circ D \circ T(\mathcal{A})$.
Si deux mots ont même résiduels, ils atteignent le même état dans l'automate des
parties de $\mathcal{B}$ en vertu du lemme. Il y a autant d'états que de
résiduels dans $D(\mathcal{B})$, ce dernier est donc minimal.

\begin{proposition}
  La complexité de cet algorithme est simplement exponentielle en le nombre
  d'états de l'automate d'entrée.
\end{proposition}

Pour une raison simple : la première fois qu'on déterminise, on peut avoir une
explosion exponentielle ; la seconde fois, c'est impossible puisque la taille de
la sortie est connue pour être celle d'un AFDC minimal !

TODO : réfléchir à la complexité précise de la déterminisation (avec taille de
l'alphabet en paramètre).

Pour avoir une minoration correspondante, on considère $L = X^{n-1}aX^*$ où $X =
\{a,b\}$ est l'alphabet. $L$ est reconnaissable par un AFD à $n+1$ états, et son
miroir a $2^n$ résiduels ! La minimisation ne fait que \emph{rajouter} un état
puits pour rendre l'automate complet.

\subsection{Union-find avec compression de chemin}

Trouvable sur CS Stack Exchange.

\[ \Phi(x) = \lambda \sum_{i=1}^n \log_2 w_i \]

On ne va étudier que Find (exercice : montrer que pour Union c'est bon).

\subsection{Tri par tas}

\subsection{Sélection en temps linéaire}

Avec une astuce pour rendre le truc plus rapide, merci Jill-Jênn !

\subsection{Réduction de 2SAT à la connexité forte}

\subsection{Rationalité des témoins en arithmétique de Presburger}

Soit $\phi(x_1,\ldots,x_k)$ une formule à $k$ variables libres de l'arithmétique
de Presburger, c'est-à-dire formée à partir des symboles $0$, $1$, $+$, $=$.
Soit $S_\phi$ son ensemble de témoins d'existence : $S_\phi = \{ (n_1, \ldots,
n_k) \in \N^k \mid \N \models \phi(n_1, \ldots, n_k) \}$. Définissons maintenant
le langage suivant :

\[ L_\phi = \left\{  \right\}\]
C'est compliqué à écrire formellement, mais ça représente juste des $k$-uplets
d'entiers codés en binaire. Comme la taille des entiers est non bornée et $k$
est fixe, on \enquote{transpose} les listes de listes de bits représentant
naturellement ces $k$-uplets pour avoir des mots sur un alphabet fini.

$L_\phi$ est un langage sur l'alphabet $\{0,1\}^k$.

\begin{proposition}
  $L_\phi$ est un langage rationnel, et de plus, on dispose d'une procédure
  effective pour construire un automate reconnaissant $L_\phi$ à partir de
  $\phi$.
\end{proposition}

\begin{corollary}
  L'arithmétique de Presburger est décidable.
\end{corollary}

\subsection{}



\section{Idées exclues}

\subsection{Unicité de la topologie de $\R$-EVT séparé en dimension finie}

(EVT = \emph{espace vectoriel topologique}, c'est-à-dire espace vectoriel muni
d'une topologie rendant continues les lois de compositions interne et externe.)

Évidemment, $\R$ est considéré avec sa topologie usuelle, et ça marche aussi
pour les $\C$-EVT.

\begin{theorem}
  Le $\R$-espace vectoriel $\R^n$ n'admet qu'une seule topologie de $\R$-EVT
  séparé, qui est la topologie produit.
\end{theorem}

\begin{remark}
  Les topologies non séparées sont obtenues comme topologie initiale d'une
  projection sur un quotient séparé et sont donc en bijection avec les
  sous-espaces vectoriels (prendre l'adhérence de $\{0\}$).
\end{remark}

Un résultat qui généralise l'équivalence des normes et clôt la question. Fait
dans les premières pages de Bourbaki, \emph{Espaces vectoriels topologiques},
dans le cadre général des corps valués non discrets.

\subsection{Lemme de Hensel, ou méthode de Newton $p$-adique}

Violemment hors-programme, dommage, c'est plus original que la méthode de Newton
usuelle et c'est de la jolie algèbre en plus… Un PDF de Keith Conrad raconte ça
super bien.

\begin{definition}
  On appelle \emph{entier $p$-adique} une suite $(a_n)_{n \in \N^*}$
  avec $a_n \in \Z/p^n\Z$ et $\pi_n(a_{n+1}) = a_n$ pour $n \in \N^*$,
  où $\pi_n : \Z/p^{n+1}\Z \to \Z/p^n\Z$ est la projection canonique.
\end{definition}
\begin{proposition}
  Les entiers $p$-adiques forment un sous-anneau de
  $\prod_{n \in \N^*} \Z/p^n\Z$, de caractéristique nulle. Cet anneau
  est noté $\Z_p$.
\end{proposition}

\begin{definition}
  Le \emph{corps des nombres $p$-adiques} $\Q_p$ est le complété de
  $\Q$ pour la distance $d_p(x,y) = p^{-v_p(x-y)}$.
\end{definition}
\begin{proposition}
  La boule fermée de centre 0 et de rayon 1 dans $\Q_p$ est un
  sous-anneau isomorphe à $\Z_p$.
\end{proposition}

\begin{lemma}[Hensel]
  Soient $f \in \Z_p[X]$ et $a \in \Z/p\Z$ tels que dans $\Z/p\Z$, on
  ait $f(a) = 0$ et $f'(a) \neq 0$. Alors il existe un unique
  $\alpha \in \Z_p$ tel que $f(\alpha) = 0$ et dont la classe dans
  $\Z/p\Z$ soit $a$.
\end{lemma}

\begin{corollary}
  Soient $f \in \Z[X]$ et $a \in \Z$ tels que $f(a) \equiv 0 \mod p$
  et $f'(a) \not\equiv 0 \mod p$. Alors pour tout $n \in \N^*$, il
  existe $\alpha \in \Z$ tel que $f(\alpha) \equiv 0 \mod p^n$ et
  $\alpha \equiv a \mod p$.
\end{corollary}

On a deux procédés d'itération légèrement différents pour relever des solutions
dans un $\Z/p^n\Z$ plus grand : l'un est la méthode de Newton, l'autre ressemble
à la preuve du théorème d'inversion locale.

\subsection{Schéma de réflexion de Kreisel}

\begin{theorem}[Kreisel]
  Soit $T$ un sous-système fini de l'arithmétique de Peano. Pour toute formule
  close $F$, AP prouve : \enquote{$F \text{ prouvable dans } T \Rightarrow F$}.
\end{theorem}
Note : dans la formule $\phi$ dont il est affirmé que $AP \vdash \phi$, à gauche
de $\Rightarrow$, on a un prédicat de prouvabilité appliqué au code de Gödel de
$F$, et à droite, on a vraiment $F$.
\begin{corollary}
  L'arithmétique de Peano prouve la cohérence de ses sous-systèmes finis, et
  n'est donc pas finiment axiomatisable.
\end{corollary}

D'abord, on peut se ramener au cas où $T = \emptyset$. Ensuite, essentiellement,
on veut montrer naïvement que $F$ prouvable $\Rightarrow$ $F$ vraie par
induction sur les règles logiques, qui préservent la vérité. Comme les formules
apparaissant dans une preuve sans coupures de $F$ sont de complexité logique
bornée par celle de $F$, on peut définir un prédicat de vérité borné qui fait
marcher ça, et de sorte que \enquote{$F$ vraie} soit équivalent à $F$.

Souci majeur : les détails sont impossibles à expliciter en 15 minutes, et la
preuve repose sur la possibilité de raisonner de façon interne dans AP, ce
pourquoi il faut déjà avoir une bonne intuition de quels principes logiques y
sont autorisés (les fonctions récursives sont définissables, l'induction
structurelle est valide…) car cela devient imbuvable sans un recours intensif à
l'agitage de mains.

Je ne connais pas d'autre référence \enquote{livre de cours} que Le Point
aveugle pour ce théorème…

\subsection{Algorithme de Hirschberg}

Une superbe astuce qui consiste à utiliser un diviser-pour-régner afin de
baisser la complexité spatiale d'un algorithme de programmation dynamique,
appliquée au problème de la distance d'édition. On utilise la programmation
dynamique pour calculer le point de délimitation entre les deux sous-problèmes
récursifs.

C'est parfaitement au programme de l'agrégation, mais c'est trop long…

\end{document}
