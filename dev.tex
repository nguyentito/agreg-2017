\documentclass[a4paper, 11pt]{article}

% Math symbols, notation, etc.
% Apparently, must be loaded earlier than mathspec?
\usepackage{amsmath}
\usepackage{amssymb}
\usepackage{stmaryrd}
\usepackage{amsthm}

% Locale/encoding with XeTeX: UTF-8 by default, use fontspec
\usepackage{unicode-math}
\usepackage{polyglossia} % Modern replacement for Babel
\setmainlanguage{french} % or {english}
\usepackage{csquotes} % guillemets

% TeX Gyre Pagella = URW Palladio (free Palatino) extended
\setmainfont[Mapping=tex-text]{TeX Gyre Pagella}
\setmathfont{Asana Math}

% Other
\usepackage{fullpage}
%\usepackage{enumerate}
%\usepackage{graphicx}

% Macros de Jill-Jênn
\def\A{{\cal{A}}}
\def\F{\mathbb{F}}
\def\Z{\mathbb{Z}}
\def\N{\mathbb{N}}
\def\Q{\mathbb{Q}}
\def\U{\mathbb{U}}
\def\K{\mathbb{K}}
\def\P{\mathbb{P}}
\def\R{\mathbb{R}}
\def\C{\mathbb{C}}
\def\L{{\cal{L}}}
\def\S{{\cal{S}}}
\def\V{{\cal{V}}}
\def\T{{\cal{T}}}
\def\O{{\cal{O}}}
\def\Fs{{\cal{F}}}
\def\Ps{{\cal{P}}}
\def\Cf{{\cal{C}}}
\def\M{\mathcal{M}}
\def\MnK{{\cal M}_n(\K)}
\def\Tr{\textnormal{Tr}}
\def\Sp{\textnormal{Sp}}
\def\Re{\textnormal{Re}}
\def\Vect{\textnormal{Vect}}
\def\car{\textnormal{car}}
\def\pgcd{\textnormal{pgcd}}
\def\ppcm{\textnormal{ppcm}}
\def\Sigmap{\mathfrak{S}}
\def\prog{\texttt{prog}}

% Ajout personnel
\def\E{\mathbb{E}}
\def\Var{\textnormal{Var}}
\def\Ker{\textnormal{Ker}}
\def\Indic{\mathbb{1}}

\newtheorem*{definition}{Définition}
\newtheorem*{example}{Exemple}
\newtheorem*{proposition}{Proposition}
\newtheorem*{theorem}{Théorème}
\newtheorem*{application}{Application}
\newtheorem*{algo}{Algorithme}
\newtheorem*{lemma}{Lemme}
\newtheorem*{remark}{Remarque}
\newtheorem*{corollary}{Corollaire}

\begin{document}

\title{Mes développements d'agrég de math(-info)}
\author{Nguyễn Lê Thành Dũng}
\date{Préparation à l'agrégation de l'ENS Paris-Saclay, 2016--2017}
\maketitle

\tableofcontents

\section{Algèbre et analyse}

\subsection{Sous-espace de dimension finie de $\Cf(\R)$ stables par translation}


\subsection{Fonction tangente et permutations zigzag}

Cf. Richard P. Stanley, \emph{A Survey of Alternating Permutations}. Le terme
\enquote{permutation alternante} pouvant induire en confusion avec le groupe
alterné, nous parlerons ici de \emph{permutation zigzag}.

\begin{theorem}[Désiré André]
La série génératrice des permutations haut-bas est $tan(z)+sec(z)$.
\end{theorem}

Merci à Paul Melotti, cf. son PDF.


\section{Algèbre}

\subsection{Automorphisme exceptionnel de $\Sigmap_6$}

Référence : à trouver (c'est dans H2G2 paraît-il).

On veut montrer que $\Sigmap_6$ admet un automorphisme extérieur\footnote{Ou
  encore que le groupoïde des ensembles de cardinal 6 admet un endofoncteur
  pleinement fidèle qui n'est pas naturellement isomorphe à l'identité.}.

\begin{remark}
  Il revient au même trouver un isomorphisme $\phi : \Sigmap_6 \to \Sigmap(E)$
  qui n'est induit par aucune bijection $\{1,\ldots,6\} \to E$.
\end{remark}
\begin{proof}
  En effet, si $E = \{1,\ldots,6\}$, l'isomorphisme (automorphisme, donc)
  $\Sigmap_6 \to \Sigmap_6$ induit par une permutation $\sigma$ de
  $\{1,\ldots,6\}$ est exactement la conjugaison par $\sigma$ (penser à l'action
  de la conjugaison sur la décomposition en cycles disjoints). On retrouve ainsi
  les automorphismes intérieurs.

  Réciproquement, si $\phi : \Sigmap_6 \to \Sigmap(E)$ n'est pas induit par une
  bijection, on a tout de même une égalité de cardinaux entraînant l'existence
  d'une bijection $f : \{1,\ldots,6\} \to E$, et alors quelle que soit $f$,
  $\phi \circ \Sigmap(f)^{-1}$ est un automorphisme extérieur. S'il s'agissait
  de la conjugaison par $\sigma$, on aurait $\phi = \Sigmap(f \circ \sigma)$
  (TODO vérifier si c'est pas $-1$).
\end{proof}

Nous allons construire une telle bijection à partir du lemme ci-dessous dont la
démonstration sera renvoyée à la fin.

\begin{lemma}
  Il existe un sous-groupe $H < \Sigmap_6$ d'indice 6 agissant transitivement
  sur $\{1,\ldots,6\}$ (pour l'action canonique de $\Sigmap_6$, bien entendu).
\end{lemma}

Posons maintenant $E = \Sigmap_6/H$ l'ensemble des classes à gauche. $\Sigmap_6$
agit dessus par translation ce qui donne un morphisme $\phi : \Sigmap_6 \to
\Sigmap(E)$. $Ker(\phi)$ est distingué dans $\Sigmap_6$, et d'indice au moins 6
(taille d'une orbite), donc c'est $\{\textnormal{id}\}$. $\phi$ est ainsi
injective, et même bijective entre groupes de même cardinal.

Si $f$ était telle que $\phi = \Sigmap(f)$, alors le stabilisateur de $H \in E$
pour l'action $\phi$ serait le groupe des permutations ayant pour point fixe
$f^{-1}(H) \in \{1,\ldots,6\}$. Or ici, le stabilisateur de $H$ contient $H$, vu
comme sous-groupe de $\Sigma_6$, qui agit transitivement donc dont les éléments
n'ont aucun point fixe commun. L'existence d'une telle $f$ est donc impossible.
$\phi$ est donc un isomorphisme qui n'est pas induit par une bijection.

\paragraph{Démonstration du lemme que nous avions délaissé} C'est assez bizarre
comme truc : on fait agir $\Sigmap_5$ sur ses 5-Sylow (au nombre de 6) par
conjugaison, c'est transitif (théorème de Sylow) et ça donne un morphisme
(injectif) de $\Sigmap_5$ dans le groupe des permutations de ces six 5-Sylow.
Sur un autre ensemble de 6 éléments (on avait dit $\{1,\ldots,6\}$ au départ),
il faut transférer par un choix arbitraire et non canonique de bijection. Ça
m'embête parce que c'est pas pratique pour s'assurer de la fonctorialité de la
procédure qui a un ensemble associe $E = \Sigmap_6/H$ correspondant.


\subsection{Théorèmes de Sylow}

Pour aller vite, utiliser la démonstration due à Serre, qu'on trouvera dans le
Perrin. C'est beaucoup plus laborieux (mais moins astucieux) dans \emph{Algebra}
de Lang.

\subsection{Réciprocité quadratique par les coniques}

Dans H2G2 tome 1.

\subsection{Théorème de l'élément primitif via le résultant}

(Cf. PDF de Paul Melotti.)

\begin{theorem}
  Si $L/K$ est une extension de corps finie séparable, alors il existe $\alpha
  \in L$ tel que $L = K(\alpha)$.
\end{theorem}

Comme on a \enquote{facilement} que $L = K(\alpha_1, \ldots, \alpha_n)$, on peut
faire une récurrence sur $n$ et il suffit pour cela de traiter le cas $n = 2$ :
$L = K(\alpha,\beta)$ avec $\alpha, \beta$ qui sont dans $L$ mais pas dans $K$.

Notons $P$ et $Q$ sont les polynômes minimaux respectifs de $\alpha$ et $\beta$
sur $K$. On cherche $\gamma$ tel que $L = K(\gamma)$, qu'on cherche de la forme
$\gamma = \alpha + t\beta$. 


\subsection{Dénombrement des polynômes irréductibles de $\F_q$}

Le résultat sur $\Z/p\Z$ permet de construire les corps finis comme corps de
rupture. Trouvable dans le poly de Pellerin. Utilise l'inversion de Möbius.

\section{Analyse}

\subsection{Théorème du minimax}

Dans \emph{Techniques of Variational Analysis}.

\subsection{Théorème du point fixe de Schauder}

\begin{theorem}
  Soit $K$ une partie compacte convexe d'un EVN, $f : K \to K$ continue. Alors
  $f$ admet au moins un point fixe.
\end{theorem}
Application : théorème d'existence de Cauchy-Peano.


\subsection{Récurrence d'une marche aléatoire via séries de Fourier}

Un théorème célèbre, avec une preuve étrangement moins célèbre et pourtant
stylée, que j'ai découverte dans le cours de processus aléatoires de Josselin
Garnier et qui est également trouvable sur Internet.

\begin{theorem}[Pólya]
  La marche aléatoire symétrique sur $\Z^d$ est \emph{récurrente} si et
  seulement si $d \leq 2$.
\end{theorem}

Précisons ce que signifie \enquote{récurrente}. Cette marche aléatoire est
définie par la suite de v.a. $S_n = X_1 + \ldots + X_n$ où les $(X_i)$ sont iid
uniformes parmi $\{\pm e_1, \ldots, \pm e_d\}$. Soit $N$ le nombre de passages à
l'origine : $N = \mathrm{Card}\left\{n \in \N \mid S_n = 0\right\}$. On dit que
la marche est récurrente quand $\E[N] = +\infty$.


\paragraph{Premières remarques} On a $\displaystyle \E[N] = \E\left[
  \sum_{n=0}^\infty \Indic_{\{S_n = 0\}} \right] =
\sum_{n=0}^\infty \P(S_n = 0)$.\\
On peut ne garder que les termes pairs dans cette somme. En effet, pour tout $n
\in \N$, on établit facilement que $\sum_{i=1}^d \langle S_n, e_i \rangle \equiv
n \mod 2$, ce qui entraîne que $S_n \neq 0$ quand $n$ est impair.\\
On cherche donc une expression pour $\P(S_n = 0)$, $n = 2m$ avec $m \in \N$.

\paragraph{Utilisation de la fonction caractéristique}

Si $\varphi$ est la fonction caractéristique de $X_1$, alors celle de $S_n$
est égale à $\varphi^n$ (somme de v.a. indépendantes).

\[ \forall x \in \R^d,\quad \varphi(x) = \E\left[ e^{i \langle x, X_1 \rangle}
  \right]
  = \sum_{j=1}^d \left( \frac{1}{2d}e^{ix_j} + \frac{1}{2d}e^{-ix_j}\right)
  = \frac{1}{d} \sum_{j=1}^d \cos(x_i) \]

Pour en déduire les probabilités recherchées on utilise :
\begin{lemma} Soit $k = (k_1, \ldots, k_d) \in \Z^d$. On a la formule des
  coefficients de Fourier :
 \[ \displaystyle \P(S_n = k) = \frac{1}{(2\pi)^d} \int_{[-\pi,\pi]^d} \varphi_n(x)e^{-i\langle k,x \rangle} \]
\end{lemma}
En fait, $\phi_n$ est une série de Fourier $d$-dimensionnelle normalement
convergente dont les coefficients sont les $\P(S_n = (k_1, \ldots, k_d))$. Ce
n'est pas anecdotique : la preuve qu'on est en train de dérouler repose
fondamentalement sur un passage au domaine de Fourier\footnote{Ici les
  \enquote{fréquences} (le dual) sont dans le tore et le \enquote{temps} (le
  primal) est discret ; on est habitués à l'autre sens, mais c'est juste
  l'involutivité de la dualité de Pontriaguine. Les anglophones parlent de
  \enquote{discrete-time Fourier transform}, à ne pas confondre avec la
  transformée de Fourier discrète d'un signal \emph{fini}.} pour convertir une
convolution de mesures en produit de fonctions, c'est complètement dans l'esprit
de l'analyse de Fourier. Mais ici, pas besoin de toute cette théorie, il suffit
de calculer :
\begin{proof}[Preuve du lemme]
  \[ \int_{[-\pi,\pi]^d} \varphi_n(x)e^{-i\langle k,x \rangle} dx =
    \int_{[-\pi,\pi]^d} \E\left[ e^{i\langle S_n, x \rangle} e^{-i\langle k,x
        \rangle}\right] dx
    = \E\left[ \int_{[-\pi,\pi]^d} e^{i\langle S_n - k, x \rangle} dx \right]
    = \E\left[ (2\pi)^d \Indic_{\{S_n = k\}} \right] \]
  où on a pu intervenir $\int$ et $\E$ grâce au théorème de Fubini appliqué à
  une fonction bornée, donc intégrable sur un produit d'espaces de mesure finie.
\end{proof}
Ainsi, en se souvenant que $\varphi$ est à valeurs réelles, ses puissances
paires sont positives (!), et
\[ (2\pi)^d \E[N] = \sum_{m=0}^{\infty} \int_{[-\pi,\pi]^d} \varphi(x)^{2m} dx
  = \int_{[-\pi,\pi]^d} \frac{dx}{1 - \varphi(x)^2}\]
par convergence monotone, l'expression finale étant valable parce que
$|\varphi(x)| < 1$ presque partout. (Si on avait gardé les termes impairs (qui
peuvent être négatifs), comme c'est le cas dans certaines références,
l'interversion série-intégrale aurait été plus dure à justifier ; il faut une
astuce dans ce cas…)

\paragraph{Étude d'une intégrale et conclusion} Il s'agit donc de savoir si
cette intégrale est finie ou non. $1/(1-\varphi^2)$ part à l'infini exactement
aux points où il y a un problème de convergence à savoir $|\varphi| = 1$ , soit
$(0,\ldots,0)$ ainsi que les $2^d$ points $(\pm \pi, \ldots, \pm \pi)$ ; et elle
est continue en-dehors de ces points. Ces derniers points sont des anti-périodes
de $\varphi$, donc des périodes de $1/(1-\varphi^2)$ : il suffit donc d'étudier
l'intégrabilité en 0, sur un voisinage $B(0,\varepsilon)$ où $0 < \varepsilon <
\pi$.

Pour $x \to 0$,
\[ 1 - \varphi(x) = \frac{1}{d} \sum_{j=1}^d (1 - \cos x_j) \sim
  -\frac{1}{d}(x_1^2 + \ldots + x_d^2) \] d'où
\[\frac{1}{1 - \varphi(x)^2} = \frac{1}{(1 + \varphi(x))(1 - \varphi(x))} \sim
  \frac{2d}{\|x\|^2} \]
Finalement, reste à étudier l'intégrabilité de $\|x\|^{-2}$. Un passage en
coordonnées polaires et le théorème de Tonelli donnent
\[ \int_{B(0,\varepsilon)} \|x\|^{-2}\,dx = \int_{]0,\varepsilon[ \times S^{d-1}}
  r^{-2} \cdot r^{d-1}\,dr\,d\omega = \textnormal{Vol}(S^{d-1}) \int_0^\varepsilon
  r^{d-3}\,dr \]
Ce qui est infini exactement quand $d < 3$, CQFD.

\paragraph{Si le temps permet…} On expliquera pourquoi $\E[N] = +\infty$ est
équivalent à presque sûrement repasser par l'origine au moins une fois. (C'est
une propriété générale des chaînes de Markov.)


\subsection{Théorème de bornitude uniforme et série de Fourier divergente}

Moralement : l'évaluation en 0 de la somme partielle de Fourier est un produit
scalaire avec un noyau de Dirichlet, dont la norme tend vers $+\infty$.

\subsection{Gaussiennes et inversion de Fourier (en dimension 1)}

Notre but est de montrer, dans le cas $d = 1$ :

\begin{theorem}[Inversion de Fourier]
  Soit $f \in L^1(\R^d)$. Si $\hat{f} \in L^1(\R^d)$, on a pour presque
  tout $x \in \R^d$ :
\[ f(x) = \frac{1}{(2\pi)^d} \int_\R \hat{f}(\xi) e^{i\xi \cdot x} \,d\xi =
  \hat{\hat{f}}(-x). \]
\end{theorem}

\paragraph{Préliminaires}
On définit la gaussienne d'écart-type $\sigma > 0$ comme $g_\sigma : x \mapsto
(\sigma\sqrt{2\pi})^{-1}\exp(-x^2/2\sigma^2)$. On a $g_\sigma \in \S(\R)$, donc
elle est intégrable tout comme toutes ses dérivées.

\begin{lemma}
  (Admis !) Pour tout $f \in L^1(\R)$, $g_\sigma * f
  \underset{\sigma \to 0}{\longrightarrow} f$ dans $L^1(\R)$.
\end{lemma}
Attention, convergence $L^p$ n'entraîne pas convergence presque partout ! Par
contre, d'une suite qui converge dans $L^p$, on peut extraire une sous-suite qui
converge vers la même limite presque partout, d'où :
\begin{corollary}
  Il existe une suite $\sigma_n$ décroissante tendant vers 0 telle que
  $g_{\sigma_n} * f \rightarrow f$ p.p.
\end{corollary}
(Rappel : toute série absolument convergente dans $L^p$ converge p.p., ça
sert dans la preuve de complétude.)

\paragraph{Transformée de Fourier de la gaussienne}
Avant d'attaquer le cas général, on commence par calculer explicitement
$\widehat{g_\sigma}$.

Une façon simple est d'utiliser le théorème de dérivation
sous le signe intégral pour montrer que c'est une solution de l'équation
différentielle $y' = -\sigma^2xy $. Comme $\widehat{g_\sigma}(0) = 1$ on peut
parachuter la solution $\sqrt{2\pi}/\sigma \cdot g_{1/\sigma}$, Cauchy-Lipschitz
linéaire assurant l'unicité.

En itérant deux fois, on a bien $\widehat{\widehat{g_{\sigma}}} = 2\pi g_\sigma$
(le signe moins a disparu car $g_\sigma$ est paire).

\paragraph{Cas général} Soit $f \in L^1(\R)$ telle que $\hat{f} \in L^1(\R)$. On
pose
\[ F_\sigma(x) = \int_\R \widehat{g_\sigma}(\xi)\hat{f}(\xi)e^{i\xi x} d\xi \]
Un calcul rapide (on écrit la formule intégrale pour $\hat{f}$, puis on utilise
Fubini) entraîne que
\[ F_\sigma(x) = \widehat{\widehat{g_\sigma}} * f(\xi) \]

On observe que $\widehat{g_\sigma} \longrightarrow 1$ ponctuellement quand
$\sigma \to 0$ ; en passant à la limite sur la suite $\sigma_n$ du lemme
préliminaire, on trouve l'égalité presque partout de la formule d'inversion de
Fourier. Pour ce faire on a besoin crucialement que $\hat{f} \in L^1(\R)$ pour
que le théorème de convergence dominée s'applique.


\subsection{Polynômes de Bernstein}

Référence : Zuily-Quéffélec.

\begin{theorem}[Bernstein]
  Soit $f \in \Cf([0,1], \R)$.
  
  Pour $n \in \N$ et $x \in \R$, on pose $B_n(x) = \E[f(X)]$ où $X \sim
  \textnormal{Binom}(n,x)$. Alors :
  \begin{enumerate}
  \item Pour tout $n \in \N$, $B_n : [0,1] \to \R$ est une fonction polynomiale.
    ($B_n$ est appelé le $n$-ième polynôme de Bernstein de $f$.)
  \item $\|B_n - f\|_\infty \underset{n \to \infty}{=} O\left(
      \omega_f(n^{-1/2}) \right)$.
  \end{enumerate}
\end{theorem}

On rappelle que
\[ \omega_f(h) = \sup_{|x-y| \leq h} |f(x) - f(y)| \]
est le \emph{module de continuité uniforme} de $f$.

\begin{corollary}[Weierstrass]
  Les fonctions polynomiales sont denses dans $\Cf([0,1])$.
\end{corollary}

Le corollaire découle directement du théorème de Heine, le point 1 du théorème
est trivial en écrivant la formule de l'espérance.

Démonstration du théorème :
\begin{enumerate}
\item À $x$ fixé, écrire $|B_n(x) - f(x)|$ comme une espérance, qu'on majore en
  utilisant $\omega_f$
\item Astuce : utiliser $\omega_f(\lambda h) \leq \lceil \lambda \rceil
  \omega_f(h)$ pour obtenir $\omega_f(n^{-1/2})$ fois un facteur probabiliste
\item Majorer l'espérance de ce facteur en faisant intervenir la variance
  (dépendante de $x$, paramètre de la loi binomiale)…
\item puis indépendamment de $x$, et c'est fini !
\end{enumerate}

Il faut bien connaître le théorème de Stone-Weierstrass plus général pour
répondre aux questions sur le développement.

\subsection{Méthode de Newton (lol)}

\subsection{Lemme de Morse}

Pour avoir une application aux formules de Taylor.

\section{Informatique}

\subsection{Minimisation par double renversement (avec complexité)}

Source : Sakarovitch.

Notons $D$ la déterminisation d'un automate, prenant en entrée un AFN et
renvoyant un AFDC, partie accessible de l'automate des parties. Notons $T$ la
transposition d'un automate, obtenue en renversant le sens des flèches. La
transposée d'un automate reconnaissant $L$ reconnaît $\tilde{L}$, langage miroir
de $L$.

\begin{algo}[Calcul d'un AFDC minimal à partir d'un AFN]
  Si $\mathcal{A}$ est un automate reconnaissant le langage $L$, $D \circ T
  \circ D \circ T(\mathcal{A})$ est un AFDC minimal reconnaissant $L$.
\end{algo}

Comme la dernière opération est une déterminisation, la sortie est bien un AFDC.
Le langage reconnu est préservé puisque $D$ le préserve et $T$ prend le miroir,
ce qui est involutif. Reste à prouver la minimalité.

\begin{lemma}[Brzozowski]
  Soit $\mathcal{B}$ un automate reconnaissant $L$, \emph{co-déterministe} et
  \emph{co-accessible} (autrement dit, dont la transposée est un AFD
  accessible). Soient $u$ et $v$ vérifiant $u^{-1}L = v^{-1}L$. Alors tout état
  atteignable en lisant $u$ l'est en lisant $v$.
\end{lemma}
\begin{proof}
  Soit $q$ atteignable à partir de $u$, $q$ est co-accessible donc il existe $w$
  atteignant l'unique état final à partir de $q$. $uw \in L$ donc $vw \in L$, et
  le seul chemin acceptant pour $vw$ doit passer par $q$ après avoir lu $v$ par
  co-déterminisime… Faire un dessin !
\end{proof}

Appliquons maintenant le lemme à $\mathcal{B} = T \circ D \circ T(\mathcal{A})$.
Si deux mots ont même résiduels, ils atteignent le même état dans l'automate des
parties de $\mathcal{B}$ en vertu du lemme. Il y a autant d'états que de
résiduels dans $D(\mathcal{B})$, ce dernier est donc minimal.

\begin{proposition}
  La complexité de cet algorithme est simplement exponentielle en le nombre
  d'états de l'automate d'entrée.
\end{proposition}

Pour une raison simple : la première fois qu'on déterminise, on peut avoir une
explosion exponentielle ; la seconde fois, c'est impossible puisque la taille de
la sortie est connue pour être celle d'un AFDC minimal !

TODO : réfléchir à la complexité précise de la déterminisation (avec taille de
l'alphabet en paramètre).

Pour avoir une minoration correspondante, on considère $L = X^{n-1}aX*$ où $X =
\{a,b\}$ est l'alphabet. $L$ est reconnaissable par un AFD à $n+1$ états, et son
miroir a $2^n$ résiduels ! La minimisation ne fait que \emph{rajouter} un état
puits pour rendre l'automate complet.

\subsection{Union-find avec compression de chemin}

Trouvable sur CS Stack Exchange.

\[ \Phi(x) = \lambda \sum_{i=1}^n \log_2 w_i \]

On ne va étudier que Find (exercice : montrer que pour Union c'est bon).

\subsection{Tri par tas}

\subsection{Sélection en temps linéaire}

Avec une astuce pour rendre le truc plus rapide, merci Jill-Jênn !

\subsection{Réduction de 2SAT à la connexité forte}

\subsection{Rationalité des témoins en arithmétique de Presburger}

Soit $\phi(x_1,\ldots,x_k)$ une formule à $k$ variables libres de l'arithmétique
de Presburger, c'est-à-dire formée à partir des symboles $0$, $1$, $+$, $=$.
Soit $S_\phi$ son ensemble de témoins d'existence : $S_\phi = \{ (n_1, \ldots,
n_k) \in \N^k \mid \N \models \phi(n_1, \ldots, n_k) \}$. Définissons maintenant
le langage suivant :

\[ L_\phi = \left\{  \right\}\]
C'est compliqué à écrire formellement, mais ça représente juste des $k$-uplets
d'entiers codés en binaire. Comme la taille des entiers est non bornée et $k$
est fixe, on \enquote{transpose} les listes de listes de bits représentant
naturellement ces $k$-uplets pour avoir des mots sur un alphabet fini.

$L_\phi$ est un langage sur l'alphabet $\{0,1\}^k$.

\begin{proposition}
  $L_\phi$ est un langage rationnel, et de plus, on dispose d'une procédure
  effective pour construire un automate reconnaissant $L_\phi$ à partir de
  $\phi$.
\end{proposition}

\begin{corollary}
  L'arithmétique de Presburger est décidable.
\end{corollary}

\subsection{}



\section{Idées exclues}

\subsection{Unicité de la topologie de $\R$-EVT séparé en dimension finie}

(EVT = \emph{espace vectoriel topologique}, c'est-à-dire espace vectoriel muni
d'une topologie rendant continues les lois de compositions interne et externe.)

Évidemment, $\R$ est considéré avec sa topologie usuelle, et ça marche aussi
pour les $\C$-EVT.

\begin{theorem}
  Le $\R$-espace vectoriel $\R^n$ n'admet qu'une seule topologie de $\R$-EVT
  séparé, qui est la topologie produit.
\end{theorem}

\begin{remark}
  Les topologies non séparées sont obtenues comme topologie initiale d'une
  projection sur un quotient séparé et sont donc en bijection avec les
  sous-espaces vectoriels (prendre l'adhérence de $\{0\}$).
\end{remark}

Un résultat qui généralise l'équivalence des normes et clôt la question. Fait
dans les premières pages de Bourbaki, \emph{Espaces vectoriels topologiques},
dans le cadre général des corps valués non discrets.

\subsection{Lemme de Hensel, ou méthode de Newton $p$-adique}

Violemment hors-programme, dommage, c'est plus original que la méthode de Newton
usuelle et c'est de la jolie algèbre en plus… Un PDF de Keith Conrad raconte ça
super bien.

\begin{definition}
  On appelle \emph{entier $p$-adique} une suite $(a_n)_{n \in \N^*}$
  avec $a_n \in \Z/p^n\Z$ et $\pi_n(a_{n+1}) = a_n$ pour $n \in \N^*$,
  où $\pi_n : \Z/p^{n+1}\Z \to \Z/p^n\Z$ est la projection canonique.
\end{definition}
\begin{proposition}
  Les entiers $p$-adiques forment un sous-anneau de
  $\prod_{n \in \N^*} \Z/p^n\Z$, de caractéristique nulle. Cet anneau
  est noté $\Z_p$.
\end{proposition}

\begin{definition}
  Le \emph{corps des nombres $p$-adiques} $\Q_p$ est le complété de
  $\Q$ pour la distance $d_p(x,y) = p^{-v_p(x-y)}$.
\end{definition}
\begin{proposition}
  La boule fermée de centre 0 et de rayon 1 dans $\Q_p$ est un
  sous-anneau isomorphe à $\Z_p$.
\end{proposition}

\begin{lemma}[Hensel]
  Soient $f \in \Z_p[X]$ et $a \in \Z/p\Z$ tels que dans $\Z/p\Z$, on
  ait $f(a) = 0$ et $f'(a) \neq 0$. Alors il existe un unique
  $\alpha \in \Z_p$ tel que $f(\alpha) = 0$ et dont la classe dans
  $\Z/p\Z$ soit $a$.
\end{lemma}

\begin{corollary}
  Soient $f \in \Z[X]$ et $a \in \Z$ tels que $f(a) \equiv 0 \mod p$
  et $f'(a) \not\equiv 0 \mod p$. Alors pour tout $n \in \N^*$, il
  existe $\alpha \in \Z$ tel que $f(\alpha) \equiv 0 \mod p^n$ et
  $\alpha \equiv a \mod p$.
\end{corollary}

On a deux procédés d'itération légèrement différents pour relever des solutions
dans un $\Z/p^n\Z$ plus grand : l'un est la méthode de Newton, l'autre ressemble
à la preuve du théorème d'inversion locale.

\subsection{Une application du déterminant de Cayley-Menger}

On trouvera des détails sur ce fameux déterminant, qui généralise la formule de
Héron, dans le Zavidovique. L'application suivante est tirée d'un petit article
de recherche (TODO ajouter référence) et a été posée aux oraux de l'ENS Lyon en
2015.

\begin{theorem}
  Soit $n \in \N^*$. Il existe $n+2$ points à distances entières impaires dans
  $\R^n$ si et seulement si $n \equiv -2 \; (\mod\,16)$.
\end{theorem}

\subsection{Les fonctions monotones sont dérivables presque partout}

Suggéré par le rapport de jury d'agrég, infaisable en 15 minutes. Ingrédient
principal : l'inégalité maximale de Hardy-Littlewood. C'est fait sur le blog de
Terence Tao, et on peut le retrouver dans l'un de ses bouquins si l'on tient à
avoir une référence utilisable le jour de l'oral.

\subsection{Schéma de réflexion de Kreisel}

\begin{theorem}[Kreisel]
  Soit $T$ un sous-système fini de l'arithmétique de Peano. Pour toute formule
  close $F$, AP prouve : \enquote{$F \text{ prouvable dans } T \Rightarrow F$}.
\end{theorem}
Note : dans la formule $\phi$ dont il est affirmé que $AP \vdash \phi$, à gauche
de $\Rightarrow$, on a un prédicat de prouvabilité appliqué au code de Gödel de
$F$, et à droite, on a vraiment $F$.
\begin{corollary}
  L'arithmétique de Peano prouve la cohérence de ses sous-systèmes finis, et
  n'est donc pas finiment axiomatisable.
\end{corollary}

D'abord, on peut se ramener au cas où $T = \emptyset$. Ensuite, essentiellement,
on veut montrer naïvement que $F$ prouvable $\Rightarrow$ $F$ vraie par
induction sur les règles logiques, qui préservent la vérité. Comme les formules
apparaissant dans une preuve sans coupures de $F$ sont de complexité logique
bornée par celle de $F$, on peut définir un prédicat de vérité borné qui fait
marcher ça, et de sorte que \enquote{$F$ vraie} soit équivalent à $F$.

Souci majeur : les détails sont impossibles à expliciter en 15 minutes, et la
preuve repose sur la possibilité de raisonner de façon interne dans AP, ce
pourquoi il faut déjà avoir une bonne intuition de quels principes logiques y
sont autorisés (les fonctions récursives sont définissables, l'induction
structurelle est valide…) car cela devient imbuvable sans un recours intensif à
l'agitage de mains.

Je ne connais pas d'autre référence \enquote{livre de cours} que Le Point
aveugle pour ce théorème…

\subsection{Algorithme de Hirschberg}

Une superbe astuce qui consiste à utiliser un diviser-pour-régner afin de
baisser la complexité spatiale d'un algorithme de programmation dynamique,
appliquée au problème de la distance d'édition. On utilise la programmation
dynamique pour calculer le point de délimitation entre les deux sous-problèmes
récursifs.

C'est parfaitement au programme de l'agrégation, mais c'est trop long…

\end{document}
