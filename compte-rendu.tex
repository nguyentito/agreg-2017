\documentclass[a4paper, 11pt]{article}

% Math symbols, notation, etc.
% Apparently, must be loaded earlier than mathspec?
\usepackage{amsmath}
\usepackage{amssymb}
\usepackage{stmaryrd}
\usepackage{amsthm}

% Locale/encoding with XeTeX: UTF-8 by default, use fontspec
\usepackage{unicode-math}
\usepackage[frenchb]{babel}
\usepackage{csquotes} % guillemets

% TeX Gyre Pagella = URW Palladio (free Palatino) extended
\setmainfont[Mapping=tex-text]{TeX Gyre Pagella}
\setmathfont{Asana Math}

% Other
\usepackage{fullpage}
%\usepackage{enumerate}
%\usepackage{graphicx}

\newtheorem*{definition}{Définition}
\newtheorem*{example}{Exemple}
\newtheorem*{proposition}{Proposition}
\newtheorem*{theorem}{Théorème}
\newtheorem*{application}{Application}
\newtheorem*{algo}{Algorithme}
\newtheorem*{lemma}{Lemme}
\newtheorem*{remark}{Remarque}
\newtheorem*{corollary}{Corollaire}

\begin{document}

\title{Compte-rendu des oraux d'agrég de math (option D)}
\author{Nguyễn Lê Thành Dũng}
\date{29 juin -- 1\up{er} juillet 2017}
\maketitle

\section*{Leçon d'informatique fondamentale (29/06)}

\paragraph{Couplage}

Leçon choisie : \textbf{915 -- Classes de complexité. Exemples.}

Alternative : \emph{913 -- Machines de Turing. Applications.}\\

Horreur et damnation, un couplage qui ne laisse aucune possibilité d'échapper
aux machines de Turing ! J'ai choisi le moindre mal, en m'appuyant sur le livre
de Perifel pour construire un plan tout à fait banal :
\begin{enumerate}
\item Modèle des machines de Turing et premiers résultats (accélération,
  hiérarchie)
\item Classes de complexité en temps (P, NP, EXP)
\item Classes de complexité en espace (L, NL, PSPACE)
\end{enumerate}
Malheureusement la rédaction du plan m'a pris trop de temps ce qui ne m'a pas
permis de bien préparer mes développements.

\paragraph{Développements proposés}

Choix du jury : \textbf{Le problème d'universalité des automates finis
  non-déterministes est PSPACE-difficile}.

Alternative : \emph{Théorème de Savitch} (sans hypothèse de constructibilité en
espace).\\

Pour résumer : on a un langage PSPACE, une machine de Turing $\mathcal{M}$ à un
ruban fonctionnant en espace $\leq p(n)$ ($p \in \mathbb{N}[X]$) qui le
reconnaît, et on veut construire à partir d'un mot $x \in \Sigma$ un automate
$\mathcal{A}_x$ sur un alphabet $\Sigma'$ tel que $L(\mathcal{A}_x) = \Sigma'^*$
ssi $\mathcal{M}$ \emph{rejette} $x$ (sachant que PSPACE est stable par
complémentaire). Soit $n = |x|$. On définit d'abord une représentation des
configurations de $\mathcal{M}$ comme des mots de $p(n)+1$ lettres ($p(n)$ cases
mémoire + un symbole état interne à la position de la tête de lecture), puis on
cherche à construire $\mathcal{A}_x$ de sorte que $\Sigma'^* \setminus
L(\mathcal{A}_x)$ représente l'ensemble des chemins acceptants dans
$\mathcal{M}$ sur l'entrée $x$.

Donc il faut que $\mathcal{A}_x$ détecte (1) les mots mal formés (ne correspondant
pas à des suites de configurations), (2) les suites avec deux configurations
consécutives qui ne peuvent pas se succéder dans un calcul légal, (3) celles qui
n'aboutissent pas sur un état acceptant. Il s'agissait de convaincre l'auditoire
que c'était faisable avec $|\mathcal{A}_x| = O(p(n))$. \\

En raison à la fois d'un manque de travail en amont sur ce développement et d'un
manque de temps durant la préparation, la présentation de cet automate a été
brouillonne et peu formelle / rigoureuse. Bref, j'ai fait exactement ce que JGL
nous a déconseillé de faire pendant toute l'année… S'ensuivirent 20 minutes de
questions sur le développement pour préciser des points, culminant en la
description formelle d'un automate détectant si les cases du ruban loin de la
tête de lecture ont été mal recopiés d'une configuration à la suivante.

\paragraph{Questions}

Après tout cela, il restait moins d'une dizaine de minutes pour un petit
exercice. Il s'agissait de réduire le problème de $k$-coloriabilité d'un graphe
à la $(k+1)$-coloriabilité. C'est facile, j'ai à peu près expédié le truc (en
essayant d'être assez formel cette fois-ci), sauf que j'ai écrit
$\mathrm{Col}(k+1) \leq_p \mathrm{Col}(k)$ au lieu du contraire. C'est avec ce
genre d'inattention qu'on se retrouve avec des remarques dans le rapport du jury
comme quoi même les bons candidats ne maîtrisent pas le sens des réductions…

\subsection*{Bonus : leçon d'info d'un camarade lyonnais (01/07)}

Je suis allé assister à ça l'après-midi après avoir fini mes oraux. La leçon
était \textbf{927 -- Exemples de preuve d'algorithme : correction, terminaison.}
Le développement présenté était la preuve de correction de Floyd--Warshall avec
l'optimisation en espace (réutilisation du même tableau à chaque itération de la
boucle for extérieure).

Il n'y a pas eu énormément de questions sur le développement, mais les exercices
n'étaient pas des plus passionnants. Des preuves de terminaison et correction
(détaillées, mais pas formalisées en logique de Hoare) ont été demandées pour le
tri fusion récursif et pour la recherche dichotomique itérative, ce qui
impliquait beaucoup de manipulations de parties entières.

\newpage

\section*{Modélisation et analyse de systèmes informatiques (30/06)}

Le texte que j'ai présenté portait sur \emph{l'arbre de Stern--Brocot}, un arbre
binaire infini où chaque rationnel strictement positif apparaît exactement une
fois, et sur des calculs opérant sur la représentations des rationnels comme
chemins dans cet arbre. L'autre texte du couplage parlait de codes correcteurs
d'erreurs. En fait, j'ai choisi parmi les deux celui qui parlait d'un sujet que
je connaissais déjà, puisque j'avais conçu un sujet du concours Prologin portant
là-dessus. Pour en savoir plus sur ce magnifique objet mathématique, voir
l'article \emph{Functional pearl -- Enumerating the rationals} puis parcourir
le graphe des références bibliographiques. On pourrait dire que j'ai de la
chance, mais en même temps pendant l'année il est souvent arrivé que je tombe
sur des textes de modélisation qui parlent de choses que je connaissais…

\paragraph{Le sujet}

En gros, on définit un système de réécriture sur $\mathbb{Q}_+^*$ :
\[ \frac{m}{n} \xrightarrow{G} \frac{m}{n-m} \quad \text{si}\; m < n
 \qquad \frac{m}{n} \xrightarrow{D} \frac{m-n}{n} \quad \text{si}\; m > n \]
Ensuite, on interprète les mots de $\{G,D\}^*$ comme les adresses des nœuds d'un
arbre binaire complet infini (chemins à partir de la racine avec $G$ = gauche,
$D$ = droite), et on étiquette le nœud d'adresse $w$ avec l'unique $r \in
\mathbb{Q}_+^*$  tel que $r \overset{w}{\longrightarrow^*} 1$.

Le texte donnait une méthode un peu compliquée pour calculer $r$ à partir de
$w$, tout en suggérant qu'il en existait une plus simple. Il proposait ensuite
un ordre sur $\{G,D\}^*$ dont il affirmait que ça correspondait à l'ordre total
sur les rationnels ; en déchiffrant, cela revenait en fait à dire que l'arbre de
Stern--Brocot est un \emph{arbre binaire de recherche}. Une seconde partie
proposait ensuite de calculer des homographies sur les rationnels à partir des
mots de $\{G,D\}^*$ correspondants, de façon \emph{paresseuse} : après avoir
regardé seulement un préfixe de l'entrée, on est déjà capable d'afficher un
préfixe de la sortie.

\paragraph{Exposé}

Mon plan était :
\begin{enumerate}
\item Définitions, énumération des rationnels (avec exercice de programmation)
\item Propriétés d'ordre (i.e.\ preuve que c'est un ABR)
\item Représentation des irrationnels
\item Calculs sur des flux dans $\mathbb{Q}_+^*$ et dans
  $\overline{\mathbb{R}_+}$
\end{enumerate}
Dans le temps imparti, j'ai pu traiter (1) et (2) proprement au tableau,
raconter (3) à l'oral vers la fin, et je n'ai pas du tout abordé (4). Je n'ai
pas fait de vraie conclusion à mon exposé ; c'était peut-être une faute, d'un
autre côté il n'y avait pas vraiment de situation de départ modélisée à laquelle
revenir…

Pour l'exercice de programmation, il était demandé d'écrire une fonction qui
calcule les nœuds du $n$-ième étage de l'arbre dans l'ordre. J'ai proposé deux
méthodes pour le faire :
\begin{itemize}
\item Méthode simple : pour obtenir le $i$-ième rationnel ($0 \leq i \leq 2^n -
  1$), on interprète $i$ comme un mot binaire (avec $G \leftrightarrow 0$ et $D
  \leftrightarrow 1$), puis on inverse le système de réécriture en partant de 1.
\item Méthode compliquée : on calcule d'abord le $n$-ième étage de \emph{l'arbre
    de Calkin--Wilf}, pour lequel il existe une formule magique calculant le
  successeur d'un rationnel pour l'ordre du parcours en largeur. (Grâce au temps
  de préparation abondant, j'ai pu retrouver cette formule dont je connaissais
  préalablement l'existence.) Le $n$-ième étage de l'arbre de Stern--Brocot s'en
  déduit par \emph{permutation à inversion de bits} (une opération qui
  intervient aussi dans la transformée de Fourier rapide).
\end{itemize}

Pour la partie (3), je suis parti de l'exemple, fourni par le texte, du nombre
d'or approché par $F_{n+1}/F_n$ avec $(F_n)$ la suite de Fibonacci, pour
proposer de représenter les irrationnels par des chemins \emph{infinis} dans
l'arbre. J'ai conclu l'exposé en disant que l'application $\{G,D\}^\omega \to
\overline{\mathbb{R}_+}$ n'est malheureusement pas bijective puisque chaque
rationnel a deux antécédents, mais qu'on pouvait s'y attendre puisque l'espace
de Cantor n'est pas homéomorphe à la droite réelle achevée. Le jury n'a pas
voulu saisir la perche topologique qui lui était tendue.

\paragraph{Questions} \emph{Complexité des algorithmes pour l'exercice de
  programmation ?} J'avais oublié de la mentionner… c'est en temps $O(n 2^n)$
pour les deux. Ils m'ont demandé ensuite si c'était possible de faire mieux. On
a une minoration $\Omega(2^n)$ ; j'ai dit que pour atteindre la borne, on
pourrait faire le parcours de l'arbre tronqué à la hauteur $n$ dans l'ordre
infixe, à condition de savoir passer d'un nœud à son fils gauche/droit en
$O(1)$, ce que je ne savais pas faire. Puis on est passés à autre chose.

J'avais parlé de \emph{récursivité terminale} en présentant mon code ; ils m'ont
demandé l'intérêt, puis m'ont posé quelques questions élémentaires sur cette
notion.

Comme je n'avais pas réfléchi à la méthode tordue du texte pour trouver un
rationnel à partir d'un mot, ils m'ont demandé une \emph{interprétation
  matricielle} des règles de réécriture, puis de rendre compte à la fois de ma
méthode et de celle du texte à partir de cette interprétation. Si j'ai
rapidement compris que la mienne était équivalente à multiplier successivement
un vecteur par des matrices, il a fallu qu'un membre du jury évoque
l'associativité du produit pour que je comprenne que celle du texte revenait à
calculer d'abord le produit matriciel en associant à gauche, puis de multiplier
le vecteur par la matrice obtenue.

On m'a demandé d'expliquer ce dont j'allais parler dans ma partie (4), en
particulier ce que je voulais dire par \enquote{flux}. J'ai dit que le fait
d'utiliser une stratégie paresseuse permettait d'opérer sur des suites infinies
et qu'on pouvait ainsi espérer calculer l'image d'un irrationnel par une
homographie.

Dernière question : \emph{connaissez-vous des langages de programmation
  paresseux ?} \enquote{Oui, j'ai pratiqué la programmation en Haskell,
  d'ailleurs, on a des suites infinies… mais tous les types sont coinductifs,
  donc on ne peut pas définir le type des suites forcément finies !}
L'examinateur qui avait posé la question avait l'air ravi d'entendre cela. Plus
tard, cependant, je me suis aperçu que c'était faux puisqu'on peut annoter des
champs comme stricts dans les déclarations de types !

\newpage

\section*{Leçon de mathématiques pour l'informatique (29/06)}

\def\F{\mathbb{F}}
\def\Z{\mathbb{Z}}
\def\GL{\mathrm{GL}}
\def\SL{\mathrm{SL}}
\def\PSL{\mathrm{PSL}}
\def\Sigmap{\mathfrak{S}}

\paragraph{Couplage}

Leçon choisie : \textbf{123 -- Corps finis. Applications.}

Alternative : \emph{223 -- Suites numériques. Convergence, valeurs d'adhérence.
  Exemples et applications.}\\

Étant passé sur la leçon \emph{Nombres premiers} pendant l'année, j'étais ravi
de pouvoir profiter de ma préparation spécifique ainsi que des remarques de
Claudine Picaronny. Mon plan était :
\begin{enumerate}
\item Rappels de théorie des corps (caractéristique, extensions finies)
\item Cartographie des corps finis (existence et unicité, inclusions, clôture
  algébrique)
\item Groupe multiplicatif $\F_q^*$, carrés (cyclicité avec applications au
  petit théorème de Fermat et au théorème de Wilson, puis les histoires de
  symbole de Legendre etc.)
\item Algèbre linéaire et bilinéaire sur $\F_q$ (groupes $\GL_n$ / $\SL_n$,
  formes quadratiques)
\item Polynômes irréductibles (Eisenstein, Berlekamp)
\end{enumerate}
Mes références étaient Perrin, Demazure (pour l'algorithme de Berlekamp) et
Caldero--Germoni, aussi connu sous le nom de H2G2 (pour les formes quadratiques
et les développements).

Pendant la défense du plan j'ai dessiné au tableau le trellis de tous les corps
finis d'une caractéristique $p$ fixée.



\paragraph{Développements proposés} Deux jolis développements tirés de H2G2 tome
premier.

Choix du jury : \textbf{Loi de réciprocité quadratique} (en comptant les points de
coniques sur $\F_q^p$).

Alternative : \emph{Non-isomorphisme exceptionnel des groupes simples
  $\PSL_3(\F_4)$ et $\PSL_4(\F_2)$}.\\

Je n'ai pas fait tenir le développement dans les 15 minutes : il m'a fallu
sauter une partie calculatoire pour arriver à la conclusion. À part ça, ça s'est
pas trop mal passé.

J'avais écrit $u \in \GL(\F_q^p)$ au tableau, un examinateur m'a demandé
\enquote{N'y a-t-il pas une notation plus canonique, que vous utilisez dans
  votre plan ?}. J'ai expliqué que je notais $\GL(\F_q^p)$ pour un groupes
d'applications linéaires et $\GL_p(\F_q)$ pour un groupe de matrices. Réaction :
\enquote{Ah, bon, d'accord.}.

Ils m'ont ensuite demandé de compléter le développement, j'ai commencé à
faire le calcul en montrant où le $(-1)^{(p-1)(q-1)/4}$ allait apparaître, ils
ont proposé de passer à autre chose.

Comme je parlais d'hyperplans affines dans le développement, ils m'ont demandé
le nombre d'hyperplans affines dans $\F_q^n$.

Sur l'autre développement, ils ont juste demandé : \enquote{Vous affirmez que
  ces deux groupes sont de même cardinal, quel est ce cardinal ? --- 20160, si
  je ne m'abuse. --- Ah, vingt mille, quand même !}. J'ai rajouté que c'était le
plus petit cardinal où l'on trouve deux groupes simples non isomorphes.

\paragraph{Questions}

Elles étaient nombreuses ; souvent, une fois que le jury était convaincu que ça
allait aboutir, on passait tout de suite à un autre exercice. Les voici, pas
forcément dans l'ordre :

\begin{itemize}
\item \emph{Que pouvez-vous dire de $\GL_2(\F_2)$ ?} C'est la même chose que
  $\PSL_2(\F_2)$, dont j'avais mis dans mon plan qu'il est isomorphe à
  $\Sigmap_3$. \emph{Démonstration ?} J'ai décrit l'action de $\GL_2(\F_2)$ sur
  $\F_2^2 \setminus \{0\}$, puis ils ont voulu que je prouve proprement la
  bijectivité du morphisme dans $\Sigmap_3$ correspondant. \emph{À quel
    sous-groupe correspond $\mathfrak{A}_3$ ?} J'ai explicité les matrices
  correspondantes.
\item \emph{Démonstration de la cyclicité de $\F_q^*$ ?} J'ai honteusement
  hésité sur ce grand classique, mais j'ai quand même fini par retrouver la
  preuve sans indication, ils étaient plutôt convaincus sans que je détaille
  tout.
\item \emph{Si un groupe fini est d'exposant 2, que peut-on dire ?} Réponse : on
  peut montrer que c'est abélien, c'est un $\F_2$-espace vectoriel, donc
  isomorphe à $(\Z/2\Z)^n$. Vu que je connaissais déjà le truc ils ne m'ont pas
  demandé la démonstration.
\item \emph{L'application $\F_p$-linéaire $u : x \in \F_{p^2} \mapsto x^p$
    est-elle diagonalisable ?} Il m'a demandé d'abord pourquoi c'était linéaire
  (réponse : c'est un automorphisme de corps qui fixe $\F_p$) et en particulier
  pourquoi ça préserve les sommes (\enquote{Quelle propriété des coefficients
    binomiaux utilisez-vous ? --- $p \mid C_p^k$}). Ensuite j'ai tout de suite
  remarqué que le polynôme minimal de $u$ était $X^2 - 1$ et le tour était joué.
  (À un moment, j'ai dit que $X^2 - 1 = (X-1)^2$ en pensant à mon développement
  sur $\PSL_4(\F_2)$ et en oubliant de préciser \enquote{en caractéristique 2}…)
\item \emph{Quelle est la classe d'équivalence de $q(x) = x^2 + xy + y^2 + z^2$
    sur $\F_5$ ?} Il suffit de calculer le discriminant et de tester si c'est un
  carré.
\item \emph{Comment déterminer les polynômes irréductibles de degré 2 sur
    $\F_2$~?} En éliminant les produits de polynômes de degré 1. Après avoir
  constaté qu'il y a avait 3 polynômes scindés je ne me suis pas rendu compte
  qu'il n'en restait qu'un seul d'irréductible…
\item \emph{Savez-vous décrire les $p$-Sylow de $\GL_n(\F_p)$ ?} Ça ne me disait
  pas grand-chose, mais après avoir calculé $v_p(|\GL_n(\F_p)|) = p^{n(n-1)/2}$,
  j'ai dit que ça évoquait des matrices antisymétriques ou triangulaires.
  L'examinateur m'a dit de partir sur cette dernière idée, et en fait à partir
  de là c'est facile. Après l'épreuve je me suis rappelé que c'était dans la
  preuve du premier théorème de Sylow dans le Perrin (qui vient apparemment de
  J.-P. Serre).
\item \emph{Exemple d'application du critère d'Eisenstein ?} L'irréductibilité
  du $p$-ième polynôme cyclotomique, en regardant $\Phi_p(X+1)$, a semblé plaire
  au jury. Sans doute aurait-il été préférable d'inclure l'exemple dans le plan…
\item \emph{Connaissez-vous d'autres preuves de la réciprocité quadratique ?}
  J'ai cité les mots-clés \enquote{somme de Gauss} et \enquote{équivalent
    asymptotique de la fonction thêta} (\emph{cf. annexe}). \enquote{Donc il en
    existe beaucoup, des démonstrations différentes de la réciprocité
    quadratique ? --- Oui, il y en a plein ! --- Et comment est-ce que ça
    s'inscrit dans la théorie du corps des classes ? --- Désolé, je ne connais
    pas grand-chose en théorie algébrique des nombres…}.
  Puis un autre membre du jury, amusé : \enquote{Je crois que ce n'est pas au
    programme de l'agrég, ça !}
\item La question pour finir : \emph{Pouvez-vous décrire tous les morphismes de
    $\PSL_4(\F_2)$ dans $\Sigmap_3$ ?} J'ai séché pendant quelques secondes,
  puis l'examinateur me dit \enquote{Je n'ai pas choisi ce groupe par hasard, il
    est dans votre plan !} et en effet une fois qu'on se souvient que c'est un
  groupe simple, la question devient, justement, simple.
\end{itemize}

Dernière remarque : le partage du temps de parole était très inégal, l'un des
membres du jury a posé la grande majorité des questions pendant qu'une autre ne
disait quasiment rien (peut-être était-ce une analyste ?).

\newpage

\subsection*{Annexe : réciprocité des sommes de Gauss et fonction thêta}

\def\N{\mathbb{N}}
\def\R{\mathbb{R}}
\def\Re{\textnormal{Re}}

Ceci est une tentative inaboutie de fabriquer un développement original pour les
leçons sur les séries et les développements asymptotiques. Elle m'aura donc
malgré tout servi à répondre à une question du jury ! Il y a une arnaque dans la
démonstration ci-dessous ; \textbf{avis aux amateurs : je suis à la recherche
  d'une solution satisfaisante}.

Référence livresque : Richard Bellman\footnote{Un analyste qui a aussi eu une
  carrière fructueuse de mathématicien appliqué : il est l'inventeur de la
  \emph{programmation dynamique}, qui est au programme de l'agrég option D en
  algorithmique, mais intervient aussi en recherche opérationnelle et théorie du
  contrôle.}, \emph{A brief introduction to theta functions}. Voir aussi
l'article d'Anders Karlsson, \emph{Applications of heat kernels on abelian
  groups: $\zeta(2n)$, quadratic reciprocity, Bessel integrals}, qui raconte
également d'autres applications fort jolies. Cependant, aucun des deux ne
fournit de preuve rigoureuse complète.\\

Dans tout ce qui suit, on fixe $p, q \in \N^*$ premiers entre eux. La
\emph{somme de Gauss} associée est :
\[S(p,q) = \sum_{k \in \Z/q\Z} \exp(-i\pi k^2 p/q) \]

Rappelons aussi que la \emph{fonction $\theta$ de Jacobi} est définie sur
$\{\Re(z) > 0\}$ par
\[ \theta : z \mapsto \sum_{n \in \Z} e^{-\pi n^2 z} \]
et vérifie la formule d'inversion (avec le prolongement analytique de
$\sqrt{\cdot}$ à $\{\Re(z) > 0\}$)
\[ \theta(1/z) = \sqrt{z}\theta(z) \]
(preuve : via formule de Poisson, fait dans le TD d'analyse de Fourier…).

On va montrer le résultat suivant, à partir duquel la loi de réciprocité
quadratique se déduit :

\begin{theorem}[Réciprocité des sommes de Gauss]
  $\sqrt{p}S(p,q) = e^{-i\pi/4}\sqrt{q}\overline{S(q,p)}$, i.e.
  \[ \frac{1}{\sqrt{q}} \sum_{k \in \Z/q\Z}
    \exp\left(\frac{-i\pi k^2 p}{q}\right) =
    \ \frac{e^{-i\pi/4}}{\sqrt{p}} \sum_{k \in \Z/p\Z}
    \exp\left(\frac{i\pi k^2 q}{p}\right) \]
\end{theorem}

Cette identité exacte peut en fait être obtenue à partir du développement
asymptotique de $\theta(x + ip/q)$ pour $x \to 0^+$.

\begin{lemma} Soit $k \in \N^*$. Quand $x \to 0^+$,
  $\displaystyle \sum_{l = 0}^{+\infty} \exp(-\pi(k+lq)^2x)
  \sim \frac{1}{2q\sqrt{x}}$.
\end{lemma}

\begin{proof}
  La fonction $f : t \mapsto \exp(-\pi(k+tq)^2x)$ est décroissante et intégrable
  sur $\R_+$, donc par comparaison série-intégrale,
 \[ \sum_{l \geq 1} f(l) \leq \int_0^{+\infty} f(t)\,dt \leq
   \sum_{l \geq 0} f(l) \quad \text{soit} \quad
   0 \leq \sum_{l \geq 0} f(l) - \int_0^{+\infty} f(t)\,dt \leq f(0) =
   e^{-\pi k^2 x} \leq 1
 \]
 Calculons l'intégrale :
 \[ \int_0^{+\infty} \exp(-\pi(k+tq)^2x)\,dt =
   \frac{1}{q\sqrt{\pi x}} \int_{k\sqrt{\pi x}}^{+\infty} e^{-u^2}\,du
    \underset{x \to 0^+}{\sim} \frac{1}{2q\sqrt{x}}
   \qquad
  (u = \sqrt{\pi x}(k + tq))
\]
\end{proof}

\begin{proposition}
  Quand $x \to 0^+$,
  $\displaystyle \theta\left( x + i\frac{p}{q} \right)
  \sim \frac{S(p,q)}{q\sqrt{x}}$.
\end{proposition}

\begin{proof}
  Par une interversion de sommes, justifiée par la sommabilité
  de la famille considérée, on obtient facilement que
  \[ \sum_{n \in \N^*} \exp\left(-\pi n^2 \left( x + i\frac{p}{q}  \right)\right)
    = \sum_{k = 1}^q \left( \sum_{l \in \N} \exp(-\pi(k+lq)^2x) \right)
    \exp\left(\frac{-i\pi k^2 p}{q} \right)
  \]
  Grâce au lemme, on sait que les sommes intérieures sont équivalentes à
  $1/2q\sqrt{x}$, donc
\[ \theta\left( x + i\frac{p}{q} \right) = 1 + 2 \sum_{n \in \N^*}
  \exp\left(-\pi n^2 \left( x + i\frac{p}{q} \right)\right)
      \underset{x \to 0^+}{\sim} \frac{S(p,q)}{q\sqrt{x}}
\]
\end{proof}

Maintenant, écrivons l'équation fonctionnelle de $\theta$ :
  \[ \theta\left( \frac{1}{x + ip/q} \right) =
    \sqrt{x + i\frac{p}{q}} \times \theta\left( x + i\frac{p}{q} \right)
    \underset{x \to 0^+}{\sim} e^{i\pi/4} \sqrt{\frac{p}{q}}
    \times \frac{S(p,q)}{q\sqrt{x}}
  \]
D'autre part, on a
  \[ \frac{1}{x - ip/q} \underset{x \to 0^+}{=}
    x\frac{q^2}{p^2} + i \frac{q}{p} + O(x^2)  \]
\emph{Arnaquons maintenant allègrement} en considérant que notre équivalent
asymptotique, valable quand $z \to ip/q$ en suivant une demi-droite parallèle à
$\R_+$, le reste en suivant un courbe qui finit par être tangente à cette
demi-droite. Alors
\[ \theta\left( \frac{1}{x + ip/q} \right)
  = \overline{\theta\left( \frac{1}{x - ip/q} \right)}
    \underset{x \to 0^+}{\sim} \frac{\overline{S(q,p)}}{p\sqrt{xq^2/p^2}}
  \]
  On voit donc apparaître du $S(q,p)$ ! En simplifiant et en comparant nos deux
  équivalents asymptotiques, on obtient l'identité arithmétique désirée.

Pour que ça marche vraiment, il faudrait étendre le cadre de validité du lemme,
en faisant une comparaison série-intégrale plus subtile, peut-être en utilisant
une formule du genre
  \[ \int_0^{+\infty} f(t)\,dt - \sum_{n=1}^{+\infty} f(n) =
    \int_0^{+\infty} (t - \lfloor t \rfloor)f'(t)\,dt
  \]
  On pourrait dire qu'en fin de compte, on obtient la loi de réciprocité
  quadratique par une comparaison série-intégrale !

\end{document}

