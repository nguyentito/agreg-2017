\documentclass[a4paper, 11pt, twocolumn]{article}

\usepackage[frenchb]{babel}
\usepackage{csquotes} % guillemets

\usepackage{amssymb,amsthm,amsmath}
\usepackage{algorithm,caption}
\usepackage{algpseudocode}
\usepackage{proof}
\usepackage{multicol}
%\usepackage{stmaryrd,icomma}
\usepackage{url}
\usepackage{pgfpages}
% \pgfpagesuselayout{2 on 1}[a4paper,landscape]
% \usepackage[hmargin=1cm,vmargin=1.5cm]{geometry}

\usepackage[landscape, margin=2cm]{geometry}

\usepackage{fontspec}

\theoremstyle{definition}
% http://tex.stackexchange.com/questions/59278/can-i-disable-italics-in-the-body-text-of-a-theorem-without-amsthm-ntheorem
\newtheorem{definition}[equation]{Définition}
\newtheorem*{definition*}{Définition}
\newtheorem{example}[equation]{Exemple}
\newtheorem{proposition}[equation]{Proposition}
\newtheorem{theorem}[equation]{Théorème}
\newtheorem{application}[equation]{Application}
\newtheorem{algo}[equation]{Algorithme}
\newtheorem{lemma}[equation]{Lemme}
\newtheorem{remark}[equation]{Remarque}
\newtheorem{corollary}[equation]{Corollaire}
\newtheorem{code}[equation]{Code}
\newcounter{n}
\newtheorem{dev}[n]{Développement}

\algtext*{EndProcedure}
\algtext*{EndWhile}
\algtext*{EndFor}
\algtext*{EndIf}
\algrenewcommand\algorithmicprocedure{\textbf{procédure}}
\algrenewcommand\algorithmicif{\textbf{si}}
\algrenewcommand\algorithmicthen{\textbf{alors}}
\algrenewcommand\algorithmicelse{\textbf{sinon}}
\algrenewcommand\algorithmicfor{\textbf{pour}}
\algrenewcommand\algorithmicwhile{\textbf{tant que}}
\algrenewcommand\algorithmicdo{\textbf{faire}}
\algrenewcommand\algorithmicreturn{\textbf{renvoyer}}

\newcommand\lesson[1]{{\noindent \LARGE \bfseries #1}\\[1pt]}
%\def\authorship{{\null \vfill \hfill \small \@author{} -- \url{agreg-cachan.fr}}}
\def\A{{\cal{A}}}
\def\F{\mathbb{F}}
\def\Z{\mathbb{Z}}
\def\N{\mathbb{N}}
\def\Q{\mathbb{Q}}
\def\U{\mathbb{U}}
\def\K{\mathbb{K}}
\def\P{\mathbb{P}}
\def\R{\mathbb{R}}
\def\C{\mathbb{C}}
\def\L{{\cal{L}}}
\def\S{{\cal{S}}}
\def\V{{\cal{V}}}
\def\T{{\cal{T}}}
\def\O{{\cal{O}}}
\def\Fs{{\cal{F}}}
\def\Ps{{\cal{P}}}
\def\Cf{{\cal{C}}}
\def\M{\mathcal{M}}
\def\MnK{{\cal M}_n(\K)}
\def\Tr{\textnormal{Tr}}
\def\Sp{\textnormal{Sp}}
\def\Re{\textnormal{Re}}
\def\Vect{\textnormal{Vect}}
\def\car{\textnormal{car}}
\def\pgcd{\textnormal{pgcd}}
\def\ppcm{\textnormal{ppcm}}
\def\Sigmap{\mathfrak{S}}
\def\prog{\texttt{prog}}
\renewcommand\vector[1]{\left(\begin{array}{c}#1\end{array}\right)}
\newcommand\enonce[3]{\{#1\}\ #2\ \{#3\}}
\newcommand\hoare[1]{\State $\textcolor{gray}{\{#1\}}$}

% J'aime pas la logique
\def\bv{\textnormal{bv}}
\def\fv{\textnormal{fv}}
\def\dom{\textnormal{dom}}
\newcommand\sem[1]{\llbracket #1 \rrbracket}
\def\seq{\longrightarrow}
\def\LK{\mathbf{LK}}

% Arbres de preuves de séquents
\usepackage{bussproofs}

\def\MM{\mathfrak{M}}
\def\mgu{\textnormal{mgu}}

\begin{document}

\author{\textsc{Nguyễn} Lê Thành Dũng}

\lesson{918 -- Systèmes formels de preuve en logique du premier ordre. Exemples.}

\paragraph{Introduction} Il s'agit d'étudier en tant qu'objets mathématiques les
démonstrations mathématiques, les énoncés à prouver étant formulés dans le
langage de la logique du premier ordre. Le caractère syntaxique formel des
preuves ainsi définies permet donc d'obtenir des théorèmes (méta)mathématiques,
mais conduit également à des applications en informatique : semi-décidabilité
théorique, démonstration automatique, assistants de preuve… Le principe étant
que l'existence d'une preuve pour une formule doit être équivalente à la
validité universelle de cette dernière, on établira à cette fin des théorèmes de
correction et complétude sur les systèmes de preuve.\\

On se fixe, dans toute cette leçon, un ensemble dénombrable de variables
$\mathcal{V}$ et un langage $\L$ composé pour chaque $k \in \N$ d'un ensemble
$\mathcal{F}_k$ et d'un ensemble $\mathcal{R}_k$ de symboles de fonctions (resp.
relations) d'arité $k$. On suppose connues les notions de terme, formule,
théorie et modèle en logique du premier ordre.

\section{Cadre général}


% \subsection{Rappels de logique du premier ordre}

% \begin{definition}
%   L'ensemble des \emph{termes} sur $\L$ est inductivement et librement engendré
%   par les variables et les symboles de fonctions. On appelle \emph{formule
%     atomique} (ou atome) une expression $P(t_1, \ldots, t_k)$ où $P \in
%   \mathcal{R}_k$ et $t_1, \ldots, t_k$ sont des termes. Les \emph{formules} sont
%   inductivement et librement engendrées par les atomes et les connecteurs
%   logiques $\land$, $\lor$, $\lnot$, $\forall$, $\exists$. Une \emph{théorie}
%   est un ensemble de formules.
% \end{definition}

% \begin{definition}
%   Une $\L$-structure ou \emph{modèle} $\MM$ est la donnée d'un ensemble $M$ et
%   d'une interprétation des symboles de fonctions (resp. relations) d'arité $k$
%   comme applications de $M^k$ dans $M$ (resp. relations $k$-aires sur $M$). Ceci
%   permet de définir inductivement, pour une formule $A$, la satisfaction de
%   $A$ par $M$, notée $\MM \models A$.
% \end{definition}

% \begin{definition}
%   Une formule $A$ est dite (universellement) \emph{valide} lorsque $\forall
%   \MM,\; \MM \models A$.
% \end{definition}

% \subsection{Systèmes de preuve}

\begin{definition}
  Supposons qu'on ait défini un ensemble de \emph{jugements}. Une \emph{règle
    d'inférence} spécifie que certaines jugements sont déductibles à partir
  d'autres : on écrit
  \begin{prooftree}
    \AxiomC{$A_1 \quad \ldots \quad A_n$}
    \UnaryInfC{$B$}
  \end{prooftree}
  lorsque le jugement $B$ se déduit à partir de $A_1, \ldots, A_n$. (On peut
  avoir $n = 0$, dans ce cas $B$ est un axiome.)
\end{definition}

\begin{definition}
  Un \emph{système de preuve} consiste en une définition des jugements ainsi que
  d'une famille de règles d'inférences associées. Une \emph{preuve} d'un
  jugement $J$ est alors un arbre dont les nœuds sont des inférences et la
  racine est $J$ ; ainsi, les jugements prouvables sont inductivement engendrés
  par les règles d'inférence. Si $J$ exprime que la formule $A$ est vraie, on
  dira également qu'il s'agit d'une preuve de $A$, et lorsqu'une telle preuve
  existe, $A$ est dite \emph{prouvable}.
\end{definition}

\begin{definition}
  Un système de preuve est dit :
  \begin{itemize}
  \item \emph{cohérent} si $\bot$ n'est pas prouvable ;
  \item \emph{correct} si toute formule prouvable est valide ;
  \item \emph{complet} si toute formule valide est prouvable ;
  \item \emph{récursif} si la vérification d'une inférence est décidable.
  \end{itemize}
\end{definition}

\begin{remark}
  Il suffit que les axiomes soient valides et que la validité soit stable par
  inférence, pour une notion de validité d'un jugement généralisant celle d'une
  formule, pour qu'un système de preuve soit correct. La complétude est souvent
  plus diffile à établir.
\end{remark}

\begin{proposition}
  Si un système de preuve est récursif, l'ensemble des preuves est récursif et
  l'ensemble des formules prouvables est récursivement énumérable.
\end{proposition}

\begin{corollary}
  À partir de l'existence d'un système de preuve correct, complet et récursif, on montre
  que l'ensemble des formules valides est récursivement énumérable.
\end{corollary}

\begin{proposition}
  Cependant, la validité d'une formule n'est pas décidable.
\end{proposition}

\section{Préliminaires sémantiques}

Les résultats de cette section serviront à démontrer des théorèmes de
complétude. On suppose ici que $\L$ contient des constantes, c'est-à-dire
$\mathcal{F}_0 \neq \emptyset$.

\begin{definition}
  On appelle \emph{univers de Herbrand} l'ensemble des termes clos (i.e. sans
  variables).
\end{definition}

\begin{proposition}
  Toute partie de l'ensemble des atomes clos définit un modèle sur l'univers de
  Herbrand qui satisfait exactement ces formules, appelé \emph{modèle de
    Herbrand}.
\end{proposition}

\begin{theorem}
  Soit $A = \exists x_1 \ldots \exists x_n\, B$ une formule existentielle ($B$
  sans quantificateur). Alors $A$ est valide si et seulement si elle est
  satisfaite par tout modèle de Herbrand.
\end{theorem}

\begin{theorem}[Théorème de Herbrand sémantique]
  Si $A = \exists x_1 \ldots \exists x_n\, B$ est une formule existentielle,
  elle est valide si et seulement s'il existe un entier $k$ et $k$ instances $B
  \sigma_1, \ldots, B \sigma_k$ de $B$ telles que $B \sigma_1 \lor \ldots \lor B
  \sigma_k$ soit valide.
\end{theorem}

\begin{dev}
  Démonstration des deux théorèmes précédents.
\end{dev}

\begin{proposition}
  Soit $A$ une formule, alors on peut construire à partir de $A$, en ajoutant
  des symboles de fonction au langage, les formules :
  \begin{itemize}
  \item (Skolemisation) $A^S$, universelle, qui est satisfaisable ssi $A^S$
    l'est ;
  \item (Herbrandisation) $A^H$, existentielle, qui est valide ssi $A$ l'est.
  \end{itemize}
\end{proposition}

% \section{Déduction naturelle}

% \begin{definition}[Déduction naturelle minimale]
%   Il s'agit d'un système de preuve dont les jugements sont de la forme $\Gamma
%   \vdash A$, où $\Gamma = \Gamma_1, \ldots, \Gamma_k$ est une liste de
%   formules et $A$ est une formule. Interprétation : la conjonction des
%   $\Gamma_i$ entraîne $A$. Les règles d'inférence sont :
%   \begin{center}
%     \AxiomC{}
%     \RightLabel{(Ax)}
%     \UnaryInfC{$\Gamma, A \vdash A$}
%     \DisplayProof
%     \qquad
%     \AxiomC{$\Gamma, A \vdash B$}
%     \RightLabel{$\Rightarrow$I}
%     \UnaryInfC{$\Gamma \vdash A \Rightarrow B$}      
%     \DisplayProof
%   \end{center}

% \end{definition}

\section{Calcul des séquents}

\begin{definition}
  Le \emph{calcul des séquents}, noté $\LK$, est un système de preuves dont les
  jugements sont de la forme $\Gamma \vdash \Delta$, où $\Gamma = \Gamma_1,
  \ldots, \Gamma_n$ et $\Delta = \Delta_1, \ldots, \Delta_m$ sont des ensembles
  de formules. Interprétation : la conjonction des $\Gamma_i$ entraîne la
  disjonction des $\Delta_j$. Les règles d'inférence sont les règles
  \emph{axiome} et \emph{coupure}
  \begin{center}
    \AxiomC{}
    \RightLabel{(Axiome)}
    \UnaryInfC{$\Gamma, A \vdash A, \Delta$}
    \DisplayProof
    \qquad
    \AxiomC{$\Gamma \vdash A, \Delta$}
    \AxiomC{$\Gamma', A \vdash \Delta'$}
    \RightLabel{(Coupure)}
    \BinaryInfC{$\Gamma, \Gamma' \vdash \Delta, \Delta'$}
    \DisplayProof
  \end{center}
  ainsi que des \emph{règles logiques} permettant d'introduire les connecteurs
  logiques (cf. annexe).
\end{definition}

\begin{example}[Paradoxe du buveur]
  La formule $\exists x . (P(x) \Rightarrow \forall y . P(y))$ est prouvable
  dans $\LK$ (preuve en annexe).
\end{example}

\begin{proposition}
  $\LK$ est correct et récursif.
\end{proposition}

\begin{definition}
  On appelle \emph{preuve sans coupure} une preuve n'utilisant pas la règle de
  coupure. Le calcul des séquents sans cette règle est noté $\LK^{-}$.
\end{definition}

\begin{proposition}[Propriété de la sous-formule]
  Toute formule apparaissant à l'intérieur une preuve sans coupure peut être
  obtenue par substitution d'une sous-formule d'une formule apparaissant dans la
  conclusion de la preuve.
\end{proposition}

\begin{theorem}[Élimination des coupures]
  Si $\Gamma \vdash_{\LK} \Delta$, alors $\Gamma \vdash_{\LK^{-}} \Delta$. De
  plus, on peut obtenir une preuve sans coupures à partir d'une preuve avec
  coupures par un algorithme effectif (système de réécriture fortement
  normalisant).
\end{theorem}

\begin{remark}
  Ceci permet de réduire l'espace de recherche des preuves dans les procédures
  de semi-décision pour la logique du premier ordre.
\end{remark}

\begin{corollary}
  $\LK$ est cohérent.
\end{corollary}

\begin{theorem}[Théorème de Herbrand syntaxtique]
  Si $A$ est sans quantificateur, $\exists x_1 \ldots \exists x_n\, A$ est
  prouvable dans $\LK$ si et seulement s'il existe un entier $k$ et $k$
  instances $A \sigma_1, \ldots, A \sigma_k$ de $A$ telles que $A \sigma_1 \lor
  \ldots \lor A \sigma_k$ soit prouvable dans $\LK$.
\end{theorem}

\begin{lemma}
  $A$ est prouvable dans $\LK$ ssi son herbrandisée $A^H$ l'est.
\end{lemma}

\begin{theorem}
  $\LK$ est complet.
\end{theorem}

\section{Résolution et programmation logique}

Dans cette section, on suppose connue la théorie de l'unification syntaxique de
termes. On rappelle que si deux termes $t$ et $u$ sont unifiables, alors il
existe un unificateur le plus général, noté $\mgu(t,u)$.

\begin{definition}
  Un \emph{littéral} est un atome ou une négation d'un atome. Une \emph{clause}
  est une disjonction de littéraux, sans quantificateurs. Elle est implicitement
  universellement quantifiée sur ses variables libres.
\end{definition}

\begin{definition}
  La \emph{résolution} est un système de preuve dont les jugements sont des
  clauses et les règles sont la \emph{résolution} et la \emph{factorisation}
  \begin{center}
    \AxiomC{$C \lor L$}
    \AxiomC{$\lnot L' \lor C'$}
    \RightLabel{Résolution}
    \BinaryInfC{$C\sigma \lor C'\sigma$}
    \DisplayProof
    \quad
    \AxiomC{$C \lor L \lor L'$}
    \RightLabel{Factorisation}
    \UnaryInfC{$C\sigma \lor L\sigma$}
    \DisplayProof
  \end{center}
  ($L$, $L'$ littéraux, et $\sigma = \mgu(L, L')$ pour les deux règles) ainsi
  qu'une règle d'échange correspondant à l'associativité et la commutativité de
  $\lor$.
\end{definition}

\begin{example}
  La réfutation de $(P(x,a) \lor P(a,x)) \land (P(y,a) \lor P(a,y))$ (cf.
  annexe) nécessite la règle de factorisation.
\end{example}

\begin{proposition}[Correction]
  Si $C$ est déduite de $C_1, \ldots, C_n$ par résolution, alors $C_1 \land
  \ldots \land C_n \models C$. En particulier, si $C$ est la clause vide $\bot$,
  alors $C_1 \land \ldots \land C_n$ est insatisfaisable.
\end{proposition}

\begin{lemma}[de relèvement]
  Si $C$ et $C'$ sont des clauses, $\hat{C}$ et $\hat{C}'$ des instances avec
  des termes clos, et $\hat{D}$ est obtenue à partir de $\hat{C}$ et $\hat{C}'$
  par la règle de résolution \emph{propositionnelle}, alors il existe $D$ dont 
  $\hat{D}$ est une instance et qui découle de $C$ et $C'$ par la règle de
  résolution au premier ordre.
\end{lemma}

\begin{theorem}[Complétude réfutationnelle]
  Si $C_1, \ldots, C_n$ est insatisfaisable, alors il est possible d'obtenir
  $\bot$ par résolution à partir de $C_1, \ldots, C_n$.
\end{theorem}

\begin{dev}
  Preuve de la complétude réfutationnelle, utilisant les modèles de Herbrand
  ainsi que le \emph{lemme de König faible} : tout arbre binaire sans branche
  infinie est fini.
\end{dev}

\begin{definition}
  Une \emph{clause de Horn} est une clause avec 0 ou 1 littéral positif. On
  parle de \emph{but} dans le premier cas, \emph{règle} dans le second cas. On
  appelle aussi \emph{fait} une clause sans littéral négatif (consistant
  uniquement d'un littéral positif).
\end{definition}

\begin{proposition}
  Les clauses de Horn sont stables par résolution.
\end{proposition}

\begin{algo}[Résolution SLD]
  On souhaite réfuter un \emph{programme logique}, composé de règles et d'un but
  $C$. Pour cela, on cherche à construire par résolution une suite de clauses de
  but $C = C_1, \ldots, C_n = \bot$, de sorte que pour tout $i$, $C_{i+1}$ soit
  obtenue par résolution de $C_i$ et d'une des clauses de règles, sur son unique
  littéral positif. L'algorithme est paramétré par une fonction de
  \emph{sélection} qui dit sur quel atome tenter d'effectuer une résolution à
  chaque étape.
\end{algo}

\begin{proposition}
  La résolution SLD est correcte et réfutationnellement complète sur les clauses
  de Horn.
\end{proposition}

\begin{theorem}
  Si $A$ est une formule existentielle, l'exécution d'un programme logique de
  but $\lnot A$ trouve une preuve par résolution qui fournit des témoins
  d'existence de $A$, ce qui permet de calculer des fonctions. C'est le principe
  du langage de programmation Prolog.
\end{theorem}

\begin{theorem}
  La programmation logique est Turing-complète : toute fonction récursive peut
  être encodée par des clauses de Horn.
\end{theorem}

\section*{Développements}

\begin{enumerate}
\item Théorème de Herbrand sémantique
\item Complétude de la méthode de résolution
\end{enumerate}

\newpage

\section*{Annexes}

\subsection*{Règles du calcul des séquents}

\subsubsection*{Connecteurs binaires}

\begin{center}

\AxiomC{$\Gamma, A \vdash \Delta$}
\UnaryInfC{$\Gamma, A \land B \vdash \Delta$}
\DisplayProof
\,
\AxiomC{$\Gamma, B \vdash \Delta$}
\UnaryInfC{$\Gamma, A \land B \vdash \Delta$}
\DisplayProof
\qquad
\AxiomC{$\Gamma \vdash A, \Delta$}
\AxiomC{$\Gamma \vdash B, \Delta$}
\BinaryInfC{$\Gamma \vdash A \land B, \Delta$}
\DisplayProof

\vspace{3mm}

\AxiomC{$\Gamma, A \vdash \Delta$}
\AxiomC{$\Gamma, B \vdash \Delta$}
\BinaryInfC{$\Gamma, A \lor B \vdash \Delta$}
\DisplayProof
\qquad
\AxiomC{$\Gamma \vdash A, \Delta$}
\UnaryInfC{$\Gamma \vdash A \lor B, \Delta$}
\DisplayProof
\,
\AxiomC{$\Gamma \vdash B, \Delta$}
\UnaryInfC{$\Gamma \vdash A \lor B, \Delta$}
\DisplayProof
\vspace{3mm}

\AxiomC{$\Gamma \vdash A, \Delta$}
\AxiomC{$\Gamma, B \vdash \Delta$}
\BinaryInfC{$\Gamma, A \Rightarrow B \vdash \Delta$}
\DisplayProof
\quad
\AxiomC{$\Gamma, A \vdash B, \Delta$}
\UnaryInfC{$\Gamma \vdash A \Rightarrow B, \Delta$}
\DisplayProof

\end{center}

\subsubsection*{Négation}

\begin{center}
\AxiomC{$\Gamma, A \vdash \Delta$}
\UnaryInfC{$\Gamma \vdash \neg A, \Delta$}
\DisplayProof
\qquad
\AxiomC{$\Gamma \vdash A, \Delta$}
\UnaryInfC{$\Gamma, \neg A \vdash \Delta$}  
\DisplayProof
\end{center}

\subsubsection*{Quantificateurs}

\begin{center}

\AxiomC{$\Gamma, A[t/x] \vdash \Delta$}
\UnaryInfC{$\Gamma, \forall x A \vdash \Delta$}
\DisplayProof
\qquad
\AxiomC{$\Gamma, A \vdash \Delta$}
\RightLabel{($x$ non libre dans $\Gamma, \Delta$)}
\UnaryInfC{$\Gamma, \exists x A \vdash \Delta$}
\DisplayProof

\vspace{3mm}

\AxiomC{$\Gamma \vdash A, \Delta$}
\RightLabel{($x$ non libre dans $\Gamma, \Delta$)}
\UnaryInfC{$\Gamma \vdash \forall x A, \Delta$}
\DisplayProof
\qquad
\AxiomC{$\Gamma \vdash A[t/x], \Delta$}
\UnaryInfC{$\Gamma \vdash \exists x A, \Delta$}
\DisplayProof

\end{center}


\subsection*{Paradoxe du buveur en calcul des séquents}

\begin{prooftree}
    \AxiomC{}
    \UnaryInfC{$P(x), P(z) \vdash P(z), \forall y . P(y)$}
    \UnaryInfC{$P(x) \vdash P(z), P(z) \Rightarrow \forall y . P(y)$}
    \UnaryInfC{$P(x) \vdash P(z), \exists x (P(x) \Rightarrow \forall y . P(y))$}
    \UnaryInfC{$P(x) \vdash \forall y . P(y), \exists x. (P(x) \Rightarrow \forall y . P(y))$}
    \UnaryInfC{$\vdash P(x) \Rightarrow \forall y . P(y), \exists x . (P(x) \Rightarrow \forall y . P(y))$}
    \UnaryInfC{$\vdash \exists x . (P(x) \Rightarrow \forall y . P(y)), \exists x (P(y) \Rightarrow \forall y . P(y))$}
    \UnaryInfC{$\vdash \exists x . (P(x) \Rightarrow \forall y . P(y))$}
  \end{prooftree}

\subsection*{Exemple de preuve par résolution}

\begin{prooftree}
    \AxiomC{$P(x,a) \lor P(a,x)$}
    \UnaryInfC{$P(a,a)$}
    \AxiomC{$\lnot P(y,a) \lor \lnot P(a,y)$}
    \UnaryInfC{$\lnot P(a,a)$}
    \BinaryInfC{$\bot$}
  \end{prooftree}


\end{document}
