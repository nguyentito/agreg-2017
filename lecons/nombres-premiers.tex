\documentclass[a5paper, 10pt]{article}

\usepackage[frenchb]{babel}
\usepackage{csquotes} % guillemets

\usepackage{amssymb,amsthm,amsmath}
\usepackage{algorithm,caption}
\usepackage{algpseudocode}
\usepackage{proof}
\usepackage{multicol}
%\usepackage{stmaryrd,icomma}
\usepackage{url}
\usepackage{pgfpages}
\pgfpagesuselayout{2 on 1}[a4paper,landscape]
\usepackage[hmargin=1cm,vmargin=1.5cm]{geometry}

\usepackage{unicode-math}

% TeX Gyre Pagella = URW Palladio (free Palatino) extended
\setmainfont[Mapping=tex-text]{TeX Gyre Pagella}
\setmathfont{Asana Math}


\theoremstyle{definition}
% http://tex.stackexchange.com/questions/59278/can-i-disable-italics-in-the-body-text-of-a-theorem-without-amsthm-ntheorem
\newtheorem{definition}[equation]{Définition}
\newtheorem*{definition*}{Définition}
\newtheorem{example}[equation]{Exemple}
\newtheorem{proposition}[equation]{Proposition}
\newtheorem{theorem}[equation]{Théorème}
\newtheorem{application}[equation]{Application}
\newtheorem{algo}[equation]{Algorithme}
\newtheorem{lemma}[equation]{Lemme}
\newtheorem{remark}[equation]{Remarque}
\newtheorem{corollary}[equation]{Corollaire}
\newtheorem{code}[equation]{Code}
\newcounter{n}
\newtheorem{dev}[n]{Développement}

\algtext*{EndProcedure}
\algtext*{EndWhile}
\algtext*{EndFor}
\algtext*{EndIf}
\algrenewcommand\algorithmicprocedure{\textbf{procédure}}
\algrenewcommand\algorithmicif{\textbf{si}}
\algrenewcommand\algorithmicthen{\textbf{alors}}
\algrenewcommand\algorithmicelse{\textbf{sinon}}
\algrenewcommand\algorithmicfor{\textbf{pour}}
\algrenewcommand\algorithmicwhile{\textbf{tant que}}
\algrenewcommand\algorithmicdo{\textbf{faire}}
\algrenewcommand\algorithmicreturn{\textbf{renvoyer}}

\newcommand\lesson[1]{{\noindent \LARGE \bfseries #1}\\[1pt]}
%\def\authorship{{\null \vfill \hfill \small \@author{} -- \url{agreg-cachan.fr}}}
\def\A{{\cal{A}}}
\def\F{\mathbb{F}}
\def\Z{\mathbb{Z}}
\def\N{\mathbb{N}}
\def\Q{\mathbb{Q}}
\def\U{\mathbb{U}}
\def\K{\mathbb{K}}
\def\P{\mathbb{P}}
\def\R{\mathbb{R}}
\def\C{\mathbb{C}}
\def\L{{\cal{L}}}
\def\S{{\cal{S}}}
\def\V{{\cal{V}}}
\def\T{{\cal{T}}}
\def\O{{\cal{O}}}
\def\Fs{{\cal{F}}}
\def\Ps{{\cal{P}}}
\def\Cf{{\cal{C}}}
\def\M{\mathcal{M}}
\def\MnK{{\cal M}_n(\K)}
\def\Tr{\textnormal{Tr}}
\def\Sp{\textnormal{Sp}}
\def\Re{\textnormal{Re}}
\def\Vect{\textnormal{Vect}}
\def\car{\textnormal{car}}
\def\pgcd{\textnormal{pgcd}}
\def\ppcm{\textnormal{ppcm}}
\def\Sigmap{\mathfrak{S}}
\def\prog{\texttt{prog}}
\renewcommand\vector[1]{\left(\begin{array}{c}#1\end{array}\right)}
\newcommand\enonce[3]{\{#1\}\ #2\ \{#3\}}
\newcommand\hoare[1]{\State $\textcolor{gray}{\{#1\}}$}

% J'aime pas la logique
\def\bv{\textnormal{bv}}
\def\fv{\textnormal{fv}}
\def\dom{\textnormal{dom}}
\newcommand\sem[1]{\llbracket #1 \rrbracket}
\def\seq{\longrightarrow}
\def\LK{\textbf{LK}}

\begin{document}

\author{\textsc{Nguyễn} Lê Thành Dũng}

\lesson{121 --- Nombres premiers. Applications.}

\section{Définitions et propriétés de base}

\begin{definition}
  Un \emph{nombre premier} est un entier naturel qui admet exactement
  deux diviseurs dans $\N$, 1 et lui-même.
\end{definition}
\begin{example}
  Les nombres premiers inférieurs à 10 sont 2, 3, 5 et 7.
\end{example}
\begin{example}
  $n^2 + n + 41$ est premier pour $n = 0, \ldots, 39$.
\end{example}
\begin{example}
  Si $2^n+1$ est premier pour $n \in \N^*$, alors $n$ est de la forme
  $2^m$ pour $m \in \N$ (et ce n'est pas une condition suffisante). Un
  tel nombre est appelé \emph{nombre premier de Fermat}.
\end{example}

\begin{theorem}
  Il existe une infinité de nombres premiers.
\end{theorem}
\begin{algo}[Crible d'Ératosthène]
  L'algorithme suivant permet d'obtenir la liste des nombres premiers
  inférieurs à $n \in \N$ :
\begin{algorithmic}
\State $\mathrm{premier}[i] \gets \mathbf{vrai}$ pour $i = 2, \ldots, n$
\For{$i = 2$ à $\lfloor \sqrt{n} \rfloor$}
\If{$\mathrm{premier}[i]$}
\For{$k = 2$ à $\lfloor n/i \rfloor$}
\State $\mathrm{premier}[ik] \gets \mathbf{faux}$
\EndFor
\EndIf
\EndFor
\end{algorithmic}
\end{algo}

\begin{lemma}[Euclide]
  Soient $p$ premier et $a, b \in \N$. Si $p \mid ab$, alors
  $p \mid a$ ou $p \mid b$.
\end{lemma}
\begin{theorem}[Théorème fondamental de l'arithmétique]
  Tout entier $n \in \N^*$ admet une \emph{décomposition en facteurs
    premiers}, c'est-à-dire s'écrit comme produit d'un multi-ensemble
  de nombres premiers, et cette décomposition est unique.
\end{theorem}
\begin{corollary}
  $\Z$ est un anneau factoriel dont les irréductibles sont les nombres
  premiers et leurs opposés.
\end{corollary}

\begin{definition}
  La \emph{valuation $p$-adique} d'un entier $n \in \N^*$, notée
  $v_p(n)$, est la multiplicité de $p$ dans la décomposition en
  facteurs premiers de $n$.
\end{definition}
\begin{proposition}
  Pour $p$ premier et $a, b \in \N^*$,
  $v_p(\pgcd(a,b)) = \min(v_p(a), v_p(b))$ et
  $v_p(\ppcm(a,b)) = \max(v_p(a), v_p(b))$.
\end{proposition}
\begin{theorem}[Formule de Legendre]
Pour $p$ premier et $n \in \N$,
$\displaystyle v_p(n!) = \sum_{i=1}^{+\infty} \left\lfloor \frac{n}{p^i} \right\rfloor$
\end{theorem}


\section{Corps $\Z/p\Z$}

\subsection{Généralités}

\begin{proposition}
  Pour $n \in \N^*$, $\Z/n\Z$ est intègre ssi $\Z/nZ$ est un corps ssi
  $n$ est premier.
\end{proposition}
\begin{remark}
  Pour $p$ premier, le corps $\Z/p\Z$ sera aussi noté $\F_p$.
\end{remark}

\begin{definition}
  La \emph{caractéristique} d'un anneau $A$, notée $\car(A)$, est
  l'entier naturel engendrant le noyau du morphisme canonique
  $\Z \to A$.
\end{definition}
\begin{proposition}
  La caractéristique d'un corps est soit 0, soit un nombre premier
  $p$. Dans ce dernier cas, le corps contient un sous-corps isomorphe
  à $\F_p$.
\end{proposition}
\begin{corollary}
  Le cardinal d'un corps fini est une puissance de sa caractéristique,
  cette dernière étant un nombre premier.
\end{corollary}
\begin{proposition}
  Dans un corps $K$ de caractéristique $p$, pour $a, b \in K$,
  ${(a+b)}^p = a^p + b^p$.
\end{proposition}
\begin{definition}
  Si $K$ est un corps de caractéristique $p$, $x \in K \mapsto x^p$
  est appelé \emph{endormorphisme de Frobenius}.
\end{definition}
\begin{theorem}[Petit théorème de Fermat]
  L'endomorphisme de Frobenius est l'identité sur $\F_p$ pour $p$
  premier.
\end{theorem}
\begin{proposition}
  Pour $p$ premier, $\F_p^*$ est un groupe cyclique.
\end{proposition}
\begin{theorem}[Wilson]
  Un entier $n \geq 2$ est premier si et seulement si
  $(n-1)! \equiv -1 \mod n$.
\end{theorem}

\subsection{Polynômes irréductibles}

\begin{lemma}
  Soit $p$ un nombre premier et $n \in \N^*$. Le corps de
  décomposition de $X^{p^n} - X$ sur $\F_p$ a un groupe des
  inversibles cyclique. Il est donc engendré par un élément primitif.
\end{lemma}
\begin{remark}
  On peut montrer que c'est l'unique corps de cardinal $p^n$ à
  isomorphisme près.
\end{remark}
\begin{proposition}
  Pour $p$ premier, $\F_p[X]$ contient au moins un polynôme
  irréductible de chaque degré non nul.
\end{proposition}

\begin{proposition}
  Tout polynôme à coefficients entiers irréductible dans $\F_p[X]$
  ($p$ premier) est irréductible dans $\Z[X]$.
\end{proposition}
\begin{theorem}[Eisenstein]
  Soit $p$ premier et $P(X) = \sum_{i=0}^n a_i X^i \in \Z[X]$. Si on a
  $p \nmid a_n$, $p \mid a_i$ pour $i < n$ et $p^2 \nmid a_0$, alors
  $P$ est irréductible dans $\Q[X]$.
\end{theorem}
\begin{remark}
  Si de plus $P$ est un polynôme primitif (les $a_i$ sont premiers
  entre eux dans leur ensemble) alors il est aussi irréductible dans
  $\Z[X]$.
\end{remark}

\subsection{Carrés modulo $p$ et formes quadratiques}

Dans cette sous-section, $p$ est un nombre premier \emph{impair} fixé.

\begin{proposition}
  La suite
  $1 \to \{\pm 1\} \to \F_p^* \overset{{(-)}^2}{\longrightarrow} \F_p^*
  \overset{{(-)}^{(p-1)/2}}{\longrightarrow} \{\pm 1\} \to 1$ est
  exacte.
\end{proposition}
\begin{corollary}
$\F_p^*/{(\F_p^*)}^2 \simeq \{\pm 1\}$
\end{corollary}
\begin{corollary}
Pour $a \in \F_p$, \(\displaystyle a^{(p-1)/2} = \begin{cases}
   0 &\text{si \(a = 0\),} \\
  +1 &\text{si \(a\) est un carré dans \(\F_p\),} \\
  -1 &\text{sinon.}
\end{cases}\)
\end{corollary}
\begin{definition}
  On définit le \emph{symbole de Legendre} de $a \in \F_p$ par
  $\displaystyle \left(\frac{a}{p}\right) = a^{(p-1)/2}$.
\end{definition}
\begin{proposition}
  $\displaystyle \left(\frac{-1}{p}\right) = 1$ ssi $p \equiv 1 \mod 4$.
\end{proposition}
\begin{application}
  Il existe une infinité de nombres premiers de la forme $4k+1$, $k \in \N^*$.
\end{application}

\begin{definition}
  Si $A$ est une matrice symétrique inversible sur un corps $K$, on
  appelle \emph{discriminant} de $A$ la classe de $\det(A)$ dans
  $K^*/{(K^*)}^2$.
\end{definition}
\begin{proposition}
  Le discriminant est un invariant pour la relation de congruence.
\end{proposition}
\begin{theorem}[Classification des formes quadratiques sur $\F_p$]
  Deux matrices symétriques inversibles sur $\F_p$ sont congruentes
  ssi elles ont même discriminant.
\end{theorem}
\begin{remark}
  Il existe donc exactement deux classes de congruence de formes
  quadratiques non dégénérées sur $\F_p$.
\end{remark}

\begin{dev}[Loi de réciprocité quadratique]
  Soient $p$ et $q$ deux nombres premiers impairs. Alors
  $\displaystyle \left(\frac{p}{q}\right)\left(\frac{q}{p}\right) =
  {(-1)}^{\frac{(p-1)(q-1)}{4}}$.
\end{dev}

\section{Anneaux $\Z/p^n\Z$ et entiers $p$-adiques}

Dans toute cette section, $p$ est un nombre premier fixé.

\begin{proposition}
  Soit $n \geq 2$ un entier naturel.
  \begin{itemize}
  \item Si $p$ est impair, ${(\Z/p^n\Z)}^\times \simeq \Z/(p-1)p^{n-1}\Z$.
  \item Si $p = 2$ est impair, ${(\Z/2^n\Z)}^\times \simeq \Z/2\Z \times \Z/2^{n-2}\Z$.
  \end{itemize}
\end{proposition}
\begin{definition}
  On appelle \emph{entier $p$-adique} une suite $(a_n)_{n \in \N^*}$
  avec $a_n \in \Z/p^n\Z$ et $\pi_n(a_{n+1}) = a_n$ pour $n \in \N^*$,
  où $\pi_n : \Z/p^{n+1}\Z \to \Z/p^n\Z$ est la projection canonique.
\end{definition}
\begin{proposition}
  Les entiers $p$-adiques forment un sous-anneau de
  $\prod_{n \in \N^*} \Z/p^n\Z$, de caractéristique nulle. Cet anneau
  est noté $\Z_p$.
\end{proposition}

\begin{definition}
  Le \emph{corps des nombres $p$-adiques} $\Q_p$ est le complété de
  $\Q$ pour la distance $d_p(x,y) = p^{-v_p(x-y)}$.
\end{definition}
\begin{proposition}
  La boule fermée de centre 0 et de rayon 1 dans $\Q_p$ est un
  sous-anneau isomorphe à $\Z_p$.
\end{proposition}

\begin{lemma}[Hensel]
  Soient $f \in \Z_p[X]$ et $a \in \Z/p\Z$ tels que dans $\Z/p\Z$, on
  ait $f(a) = 0$ et $f'(a) \neq 0$. Alors il existe un unique
  $\alpha \in \Z_p$ tel que $f(\alpha) = 0$ et dont la classe dans
  $\Z/p\Z$ soit $a$.
\end{lemma}
\begin{remark}
  Il s'agit de l'analogue $p$-adique de la méthode de Newton sur $\R$.
\end{remark}
\begin{corollary}
  Soient $f \in \Z[X]$ et $a \in \Z$ tels que $f(a) \equiv 0 \mod p$
  et $f'(a) \not\equiv 0 \mod p$. Alors pour tout $n \in \N^*$, il
  existe $\alpha \in \Z$ tel que $f(\alpha) \equiv 0 \mod p^n$ et
  $\alpha \equiv a \mod p$.
\end{corollary}

\section{Théorie des nombres analytique et répartition}

\begin{definition}
  La \emph{fonction zêta de Riemann} est le prolongement analytique à
  $\C \setminus \{1\}$ de la fonction définie pour $Re(s) > 1$ par
$\displaystyle \zeta(s) = \sum_{n=1}^\infty \frac{1}{n^s}$.
\end{definition}
\begin{theorem}[Formule du produit eulérien]
  Pour $Re(s) > 1$,
  $\displaystyle \frac{1}{\zeta(s)} = \prod_{\text{\(p\) premier}} \left(1 - \frac{1}{p^s}\right)$.
\end{theorem}
\begin{corollary}
  $\displaystyle \sum_{\text{\(p\) premier}} \frac{1}{p}$ est une série divergente.
\end{corollary}

Les quatre théorèmes suivants sont admis sans démonstration.

\begin{theorem}[Mertens]
  Pour $n \in \N^*$, $\displaystyle \sum_{p \leq n} \frac{1}{p} = \ln \ln n + O(1)$.
\end{theorem}

\begin{definition}
  La \emph{fonction de compte des nombres premiers} est définie pour
  $x \in \R$ par $\pi(x) = \mathrm{Card}(\{p \leq x \mid \text{\(p\) premier}\})$.
\end{definition}
\begin{theorem}[Postulat de Bertrand]
  Pour tout $n \in \N^*$, $\pi(2n) - \pi(n) \geq 1$.
\end{theorem}
\begin{theorem}[\enquote{Th.\ des nombres premiers}, Hadamard -- de la Vallée Poussin]
  $\displaystyle \pi(x) \underset{x \to \infty}{\sim} \frac{x}{\log x}$
\end{theorem}

\begin{theorem}[Dirichlet]
  Si $a, r \in \N^*$ sont premiers entre eux, alors il existe une
  infinité de nombres premiers de la forme $a+nr$, $n \in \N$.
\end{theorem}
\begin{remark}
  Le cas $a = 1$ peut être prouvé de façon purement algébrique.
\end{remark}

\section{Applications}

\subsection{Cryptographie}

\begin{theorem}[Restes chinois]
  Si $m$ et $n$ sont deux entiers premiers entre eux, alors
  $\Z/(mn)\Z \simeq \Z/m\Z \times \Z/n\Z$.
\end{theorem}
\begin{corollary}
  Si $p$ et $q$ sont deux nombres premiers distincts,
  ${(\Z/(pq)\Z)}^\times \simeq \F_p^* \times \F_q^* \simeq \Z/(p-1)\Z
  \times \Z/(q-1)\Z$.
\end{corollary}
\begin{corollary}[Cas particulier du théorème d'Euler]
  Si $e, d \in \N^*$ vérifient $ed \equiv 1 \mod (p-1)(q-1)$, alors
  pour tout $x \in \Z/(pq)\Z$, $(x^e)^d = x^{ed} = x$.
\end{corollary}

\begin{application}[Chiffrement RSA]
  On génère aléatoirement $p, q$ premiers et $e \in \N^*$ et on
  calcule un inverse $d$ de $e$ modulo $(p-1)(q-1)$. Soit $n = pq$ ;
  les messages à chiffrer seront représentés comme des éléments de
  $\Z/nZ$. Alors
  \begin{itemize}
  \item la \emph{clé publique} est $(n,e)$ ;
  \item la \emph{clé privée} est $(n,d)$.
  \end{itemize}
  Pour chiffrer un message clair $m \in \Z/n\Z$, on calcule
  $c = m^e \mod n$ en utilisant la clé publique. L'opération inverse
  de déchiffrement d'un chiffré $c \in \Z/n\Z$ s'obtient par
  $m = c^d \mod n$ si l'on dispose de la clé privée.
\end{application}

\begin{proposition}[Sécurité du chiffrement RSA]
  Casser le chiffrement RSA revient à calculer des racines $e$-ièmes
  modulaires, problème conjecturé difficile à résoudre.
\end{proposition}
\begin{remark}
  Ce problème est facile à résoudre si l'on sait calculer la
  factorisation de $n$ en nombres premiers.
\end{remark}

\subsection{Théorie des groupes finis}

\begin{definition}
  Si $p$ est un nombre premier, on appelle \emph{$p$-groupe} un groupe
  dont l'ordre est une puissance de $p$.
\end{definition}
\begin{proposition}
  Le centre d'un $p$-groupe non trivial est non trivial.
\end{proposition}
\begin{corollary}
  Un groupe d'ordre $p^n$ ($p$ premier, $n \in \N$) contient des
  sous-groupes d'ordre $p^i$ pour $i = 0, \ldots, n$.
\end{corollary}
\begin{corollary}
  Pour $p$ premier, le seul $p$-groupe simple est $\Z/p\Z$.
\end{corollary}

\begin{definition}
  Si $p$ est un diviseur premier de l'ordre d'un groupe fini $G$, un
  \emph{$p$-sous-groupe de Sylow} de $G$ (ou \emph{$p$-Sylow}) est un
  sous-groupe de $G$ d'ordre $p^{v_p(|G|)}$.
\end{definition}
\begin{lemma}
  Soient $G$ un groupe, $p$ un diviseur premier de $|G|$, $H \leq G$
  un sous-groupe quelconque, et $S \leq G$ un $p$-Sylow. Alors il
  existe $a \in G$ tel que $aSa^{-1} \cap H$ soit un $p$-Sylow de $H$.
\end{lemma}
\begin{dev}[Théorèmes de Sylow]
  Soit $G$ un groupe d'ordre $p^n m$ ($p$ premier, $n, m \in
  \N^*$). Alors
  \begin{enumerate}
  \item $G$ contient un $p$-Sylow ;
  \item tout $p$-sous-groupe de $G$ est contenu dans un $p$-Sylow ;
  \item les $p$-Sylow sont tous conjugués dans $G$ ;
  \item leur nombre divise $m$ et est congru à 1 modulo $p$.
  \end{enumerate}
\end{dev}

\begin{application}
  $\C$ est algébriquement clos. (Preuve algébrique par la théorie de
  Galois.)
\end{application}

\section{Références}

\begin{itemize}
\item Serre, \emph{Cours d'arithmétique}
\item Perrin, \emph{Cours d'algèbre}
\item Caldero et Germoni, \emph{Histoires hédonistes de groupes et de géométries, tome 1}
\end{itemize}

\end{document}

